%!TEX TS-program = xelatex
\documentclass[12pt, a4paper]{article}

% Этот шаблон документа разработан в 2014 году
% Данилом Фёдоровых (danil@fedorovykh.ru) 
% для использования в курсе 
% <<Документы и презентации в \LaTeX>>, записанном НИУ ВШЭ
% для Coursera.org: http://coursera.org/course/latex .
% Исходная версия шаблона --- 
% https://www.writelatex.com/coursera/latex/5.3

% В этом документе преамбула

%%% Работа с русским языком
\usepackage{cmap}					% поиск в PDF
\usepackage{mathtext} 				% русские буквы в формулах
\usepackage[T2A]{fontenc}			% кодировка
\usepackage[utf8]{inputenc}			% кодировка исходного текста
\usepackage[english,russian]{babel}	% локализация и переносы
\usepackage{indentfirst}
\frenchspacing

\renewcommand{\epsilon}{\ensuremath{\varepsilon}}
\renewcommand{\phi}{\ensuremath{\varphi}}
\renewcommand{\kappa}{\ensuremath{\varkappa}}
\renewcommand{\le}{\ensuremath{\leqslant}}
\renewcommand{\leq}{\ensuremath{\leqslant}}
\renewcommand{\ge}{\ensuremath{\geqslant}}
\renewcommand{\geq}{\ensuremath{\geqslant}}
\renewcommand{\emptyset}{\varnothing}

%%% Дополнительная работа с математикой
\usepackage{amsmath,amsfonts,amssymb,amsthm,mathtools} % AMS
\usepackage{icomma} % "Умная" запятая: $0,2$ --- число, $0, 2$ --- перечисление

%% Номера формул
%\mathtoolsset{showonlyrefs=true} % Показывать номера только у тех формул, на которые есть \eqref{} в тексте.
%\usepackage{leqno} % Нумереация формул слева

%% Свои команды
\DeclareMathOperator{\sgn}{\mathop{sgn}}

%% Перенос знаков в формулах (по Львовскому)
\newcommand*{\hm}[1]{#1\nobreak\discretionary{}
{\hbox{$\mathsurround=0pt #1$}}{}}

%%% Работа с картинками
\usepackage{graphicx}  % Для вставки рисунков
\graphicspath{{images/}{images2/}}  % папки с картинками
\setlength\fboxsep{3pt} % Отступ рамки \fbox{} от рисунка
\setlength\fboxrule{1pt} % Толщина линий рамки \fbox{}
\usepackage{wrapfig} % Обтекание рисунков текстом

%%% Работа с таблицами
\usepackage{array,tabularx,tabulary,booktabs} % Дополнительная работа с таблицами
\usepackage{longtable}  % Длинные таблицы
\usepackage{multirow} % Слияние строк в таблице

%%% Теоремы
\theoremstyle{plain} % Это стиль по умолчанию, его можно не переопределять.
\newtheorem{theorem}{Теорема}[section]
\newtheorem{proposition}[theorem]{Утверждение}
 
\theoremstyle{definition} % "Определение"
\newtheorem{corollary}{Следствие}[theorem]
\newtheorem{problem}{Задача}[section]
 
\theoremstyle{remark} % "Примечание"
\newtheorem*{nonum}{Решение}

%%% Программирование
\usepackage{etoolbox} % логические операторы

%%% Страница
\usepackage{extsizes} % Возможность сделать 14-й шрифт
\usepackage{geometry} % Простой способ задавать поля
	\geometry{top=25mm}
	\geometry{bottom=35mm}
	\geometry{left=35mm}
	\geometry{right=20mm}
 %
%\usepackage{fancyhdr} % Колонтитулы
% 	\pagestyle{fancy}
 	%\renewcommand{\headrulewidth}{0pt}  % Толщина линейки, отчеркивающей верхний колонтитул
% 	\lfoot{Нижний левый}
% 	\rfoot{Нижний правый}
% 	\rhead{Верхний правый}
% 	\chead{Верхний в центре}
% 	\lhead{Верхний левый}
%	\cfoot{Нижний в центре} % По умолчанию здесь номер страницы

\usepackage{setspace} % Интерлиньяж
%\onehalfspacing % Интерлиньяж 1.5
%\doublespacing % Интерлиньяж 2
%\singlespacing % Интерлиньяж 1

\usepackage{lastpage} % Узнать, сколько всего страниц в документе.

\usepackage{soul} % Модификаторы начертания

\usepackage{hyperref}
\usepackage[usenames,dvipsnames,svgnames,table,rgb]{xcolor}
\hypersetup{				% Гиперссылки
    unicode=true,           % русские буквы в раздела PDF
    pdftitle={Заголовок},   % Заголовок
    pdfauthor={Автор},      % Автор
    pdfsubject={Тема},      % Тема
    pdfcreator={Создатель}, % Создатель
    pdfproducer={Производитель}, % Производитель
    pdfkeywords={keyword1} {key2} {key3}, % Ключевые слова
    colorlinks=true,       	% false: ссылки в рамках; true: цветные ссылки
    linkcolor=red,          % внутренние ссылки
    citecolor=black,        % на библиографию
    filecolor=magenta,      % на файлы
    urlcolor=cyan           % на URL
}

\usepackage{csquotes} % Еще инструменты для ссылок

%\usepackage[style=authoryear,maxcitenames=2,backend=biber,sorting=nty]{biblatex}

\usepackage{multicol} % Несколько колонок

\usepackage{tikz} % Работа с графикой
\usepackage{pgfplots}
\usepackage{pgfplotstable}




% Специальный пакет для оформления задач! 
\newtheorem{problem}{Задача}

\usepackage{answers}
\Newassociation{sol}{solution}{solution_file}
% sol --- имя окружения внутри задач
% solution --- имя окружения внутри solution_file
% solution_file --- имя файла в который будет идти запись решений


\begin{document}

% Открываем файл, куда будут записываться решения. 
\Opensolutionfile{solution_file}[all_solutions]


\begin{problem}
В разложении \[\left(\sqrt{x} + \frac{1}{\sqrt[3]{2}}\right)^n\] коэффициент пятого члена относится к коэффициенту третьего члена, как 7 к 2. Найти коэффициент перед $x$ в первой степени.
\begin{sol}

\end{sol}
\end{problem}



\begin{problem}
Сколько рациональных членов содержится в разложении \[(\sqrt{2} + \sqrt[3]{3})^{100} \]
\begin{sol}

\end{sol}
\end{problem}



\begin{problem}
Дан многочлен \[x (2-3x)^5 + x^3 (1 + 2x^2)^7 - x^4 (3 + 2x^3)^9.\] Найти коэффициент члена, содержащего $x^5$, не раскрывая скобок.
\begin{sol}

\end{sol}
\end{problem}



\begin{problem}
Найти все рациональные члены разложения \[\left( \sqrt[3]{2} - \frac{1}{\sqrt{2}} \right),\] не выписывая иррациональные.
\begin{sol}

\end{sol}
\end{problem}



\begin{problem}
В разложении \[\left( x \sqrt{x} - \frac{1}{x^4} \right)^n\] биномиальный коэффициент третьего члена на 44 больше коэффициента второго члена. Найти свободный член.
\begin{sol}

\end{sol}
\end{problem}



\begin{problem}
Найти член разложения \[\left( \frac{b \sqrt{b}}{a} - \frac{a}{\sqrt[3]{b}} \right)^{14},\] содержащий $b^{10}$.
\begin{sol}

\end{sol}
\end{problem}



\begin{problem}
Найти следующие суммы
\begin{itemize}
\item $C_n^0 + C_n^1 + C_n^2 + \ldots + C_n^n$
\item $1 - C_n^1 + C_n^2 - C_n^3 + \ldots + (-1)^n C_n^n$
\item $C_n^1 + C_n^3 + C_n^5 + \ldots + C_n^{n-1}$, где $n$ --- чётное.
\item $C_n^0 + 2C_n^1 + 2^2 C_n^2 + \ldots + 2^n C_n^n$
\item $C_n^1 + 2C_n^2 + 3C_n^3 + \ldots + n C_n^n$
\end{itemize}
\begin{sol}

\end{sol}
\end{problem}



\begin{problem}
На дереве висит 10 различных яблок. Сколькими способами можно сорвать нечётное количество яблок?
\begin{sol}

\end{sol}
\end{problem}



\begin{problem}
Пусть $A = \{a_1,a_2, \ldots a_6\}$. Чему равна мощность списка всех подмножеств множества $A$?  Сколькими способами можно 
выбрать подмножество из чётного числа элементов?
\begin{sol}

\end{sol}
\end{problem}



\begin{problem}
Верно ли, что $(1 + x)^n + (1 - x)^n \le 2^n$ при $ n \ge 2$ и $|x| \le 1$ 
\begin{sol}

\end{sol}
\end{problem}




% Закрываем файл, куда мы записывали решения и вставляем его в конце списка задач. 
\Closesolutionfile{solution_file}

% Вставляем решения. Можно их не вставлять или настроить пакет так, чтобы они шли непосредственно после каждой задачи.
% \begin{solution}{1}
\end{solution}
\begin{solution}{3}
\end{solution}
\begin{solution}{4}
\end{solution}



\end{document}
