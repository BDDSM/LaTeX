\documentclass[12pt, a4paper]{article}  
\usepackage{amsmath,amsfonts,amssymb,amsthm,mathtools} 

\usepackage{fontspec}         
\setmainfont{Arial}  
\defaultfontfeatures{Mapping=tex-text}

\newfontfamily{\cyrillicfonttt}{Arial}
\newfontfamily{\cyrillicfont}{Arial}
\newfontfamily{\cyrillicfontsf}{Arial}

\usepackage{unicode-math}    
\setmathfont{Phorssa}     

\usepackage{polyglossia}      
\setdefaultlanguage{russian}  
\setotherlanguage{english}    

\author{Маликова Ольга} 
\title{Домашнее задание 2}
\date{\today}

\begin{document} 

\maketitle

\section{Упражнение 3}
\subsection{Записка}

\Large{\fontspec{Phorssa}{Филипп, когда придумываешь домашние задания, помни : если ты задашь что-то с чем я не смогу справиться, я съем твою печень с тушеными бобами и чудным кьянти. \\
\begin{flushright}
Ганибал Лектер
\end{flushright}}

\newpage
\subsection{Формулы}

\fontspec{Arial}{Некоторые математический символы не отображаются из-за того, что их изначально нет в таблице символов для этого шрифта (например знак интеграла). Также не отображаются буквы греческого алфавита.}

\begin{equation}\label{eq:1}
\sin{\alpha}\pm \sin{\beta} = 2 \cdot \sin{\dfrac{\alpha \pm \beta}{2}} \cdot \cos{\dfrac{\alpha \mp \beta}{2}}
\end{equation}

\begin{equation}
\int_0^{\infty} e^{-x^2} dx = \frac{\pi}{2}\label{eq:1} 
\end{equation}

\end{document} 

