\documentclass[10pt, a4paper]{article}


%%%%%%%%%% Програмный код %%%%%%%%%%
\usepackage{minted}
% Включает подсветку команд в программах!
% Нужно, чтобы на компе стоял питон, надо поставить пакет Pygments, в котором он сделан, через pip.

% Для Windows: Жмём win+r, вводим cmd, жмём enter. Открывается консоль.
% Прописываем easy_install Pygments
% Заходим в настройки texmaker и там прописываем в PdfLatex:
% pdflatex -shell-escape -synctex=1 -interaction=nonstopmode %.tex


%%%%%%%%%% Математика %%%%%%%%%%
\usepackage{amsmath,amsfonts,amssymb,amsthm,mathtools}
%\mathtoolsset{showonlyrefs=true}  % Показывать номера только у тех формул, на которые есть \eqref{} в тексте.
%\usepackage{leqno} % Нумерация формул слева


%%%%%%%%%%%%%%%%%%%%%%%% Шрифты %%%%%%%%%%%%%%%%%%%%%%%%%%%%%%%%%
\usepackage{fontspec}         % пакет для подгрузки шрифтов
\setmainfont{Arial}   % задаёт основной шрифт документа

% Команда, которая нужна для корректного отображения длинных тире и некоторых других символов.
\defaultfontfeatures{Mapping=tex-text}

% why do we need \newfontfamily:
% http://tex.stackexchange.com/questions/91507/
\newfontfamily{\cyrillicfonttt}{Arial}
\newfontfamily{\cyrillicfont}{Arial}
\newfontfamily{\cyrillicfontsf}{Arial}

\usepackage{unicode-math}     % пакет для установки математического шрифта
\setmathfont{Asana Math}      % шрифт для математики
% \setmathfont[math-style=ISO]{Asana Math}


\usepackage{polyglossia}      % Пакет, который позволяет подгружать русские буквы
\setdefaultlanguage{russian}  % Основной язык документа
\setotherlanguage{english}    % Второстепенный язык документа


%%%%%%%%%% Работа с картинками %%%%%%%%%
\usepackage{graphicx}                  % Для вставки рисунков
\usepackage{graphics}
\graphicspath{{images/}{pictures/}}    % можно указать папки с картинками
\usepackage{wrapfig}                   % Обтекание рисунков и таблиц текстом



%%%%%%%%%% Графика и рисование %%%%%%%%%%
\usepackage{tikz, pgfplots}  % язык для рисования графики из latex'a


%%%%%%%%%% Гиперссылки %%%%%%%%%%
\usepackage{xcolor}              % разные цвета

% Два способа включить в пакете какие-то опции:
%\usepackage[опции]{пакет}
%\usepackage[unicode,colorlinks=true,hyperindex,breaklinks]{hyperref}



\usepackage{todonotes} % для вставки в документ заметок о том, что осталось сделать
% \todo{Здесь надо коэффициенты исправить}
% \missingfigure{Здесь будет Последний день Помпеи}
% \listoftodos --- печатает все поставленные \todo'шки


% Разные мелочи для русского языка из пакета babel
\setkeys{russian}{babelshorthands=true}

\begin{document}
\title{(Homework_4)}
\date{\today}
\author{(Liza Golovanova)}
\section{Задания на minted}
\subsection{Интерактивный Python}
\begin{minted}[mathescape,
               linenos,
               numbersep=4pt,
               frame=lines,
               framesep=2.5mm]{python}
name = input("Как Вас зовут? ")
print("Привет,", name)
\end{minted}

\subsection{Фибоначчи, привет!}
\begin{minted}[mathescape,
               linenos,
               numbersep=4pt,
               frame=lines,
               framesep=2.5mm]{python}
a, b = 0, 1
while b<100:
     print (b, end=' ')
     a, b=b, a+b
\end{minted}

\subsection{Сортировка пузырьком}
\begin{minted}[mathescape,
               linenos,
               numbersep=4pt,
               frame=lines,
               framesep=2.5mm]{python}
def bubble_sort(arrayToSort):
    a = arrayToSort
    for i in range(len(a),0,-1):
        for j in range(1, i):
            if a[j-1] > a[j]:
                tmp = a[j-1]
                a[j-1] = a[j]
                a[j] = tmp
                print (a)
    return a
ary = [5, 0, 10, 4, 1, 5, 8, 4, 3, 12, 41]
print (bubble_sort(ary))
\end{minted}
\end{document}