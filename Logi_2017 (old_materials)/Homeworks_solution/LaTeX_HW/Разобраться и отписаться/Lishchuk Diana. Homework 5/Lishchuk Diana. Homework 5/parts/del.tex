% !TEX root = ../main_file.tex
\chapter{Эмпирические методы оценки влияния банковского сектора на экономический рост}

\begin{table}[H]
	\caption{Обзор эмпирических работ}\label{tab}
	\centering
	\begin{tabularx}{\textwidth}
		{|p{0.15\linewidth}|p{0.1\linewidth}|p{0.15\linewidth}|p{0.3\linewidth}|X|}\hline
		Авторы и \newline название работы & Метод оценивания & Банковский показатель & Влияние & Используемые контрольные переменные  \\\hline
		Fidrmuc J.,\newline Fungáčová Z., \newline Weill L.\newline <<Does bank liquidity creation contribute to economic growth?>> & МНК с фиксированным эффектом & Банковская ликвидность & Имея достаточное количество ликвидности, банки могут выдавать больше кредитов домохозяйствам и фирмам. Что приведёт к тому, что на рынке будет совершаться больше операций между экономическими агентами, фирмы вкладывают средства в инновации и расширения производства, что, в свою очередь, увеличивает производительность, выпуск и экономический рост & Инфляция, рассчитанная на основе индекса цен производителей\\\hline
		King R. G., Levine R  <<Finance and growth: Schumpeter might be right>>  &МНК& Доля выданных кредитов частному сектору банками в общем объёме выданных кредитов  банками и Банком России & Чем больше кредитов выдают банки, тем больше денег у фирм и домохозяйств, которые могут их выгодно вложить в различные проекты, которые направлены на улучшение способов производства, сокращение издержек и увеличению производительности. Реализация данных проектов будет способствовать развитию и росту экономики &  Логарифм первоначального дохода; \newline  Логарифм уровня среднего образования \newline Соотношение объёма торговли к ВВП;  \newline Соотношение государственных расходов к ВВП;  \newline Средний уровень инфляции. \\\hline
		
	\end{tabularx}
\end{table}

\begin{table}[H]
	\begin{tabularx}{\textwidth}
		{|p{0.15\linewidth}|p{0.1\linewidth}|p{0.15\linewidth}|p{0.3\linewidth}|X|}
		\multicolumn{5}{r}{\normalsize Окончание таблицы \ref{tab}}\\\hline
		М. Столбова \newline «Causality between credit depth and economic growth: Evidence from 24 OECD countries» & Тесты Грейнджера и Вальда для авторегрессий& Доля объёма внутренних кредитов, выданных частному сектору, к ВВП  &Чем больше кредитов выдают частному сектору, тем больше денежных средств у населения. Спрос у населения растёт, т.е. растёт его потребление, что способствует росту ВВП& Отношение валовых внутренних сбережений к ВВП; \newline Отношение общего объёма торговли к ВВП. \\\hline
		Ahmad \newline Malik «Financial sector development and economic growth» & ОММ & натуральный логарифм доли объёма кредитов, выданных банками частному сектору в общем объёме выданных кредитов & Коэффициенты при объёме частных кредитов и соотношение объёма кредитов коммерческих банка к общему объёму выданных  кредитов банков и ЦБ положительны и значимы.  Это объясняется тем, что чем больше кредитов могут выдать коммерческие банки частному сектору, тем больше средств у населения и частных предпринимателей, которые могут их выгодно вложить или потратить на собственное потребление, что в конечном итоге стимулирует деятельность экономики и усиливает её рост & Натуральный логарифм объема частных кредитов к ВВП \newline Натуральный логарифм уровня грамотного населения \newline Натуральный логарифм объёма государственных расходов на ВВП \newline Натуральный логарифм индекса потребительских цен \newline Натуральный логарифм общего объема торговли\\\hline
	\end{tabularx}
\end{table}

\missingfigure{эконометрические работы	}
