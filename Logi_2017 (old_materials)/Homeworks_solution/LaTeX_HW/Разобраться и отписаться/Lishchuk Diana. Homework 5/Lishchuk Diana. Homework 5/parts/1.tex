% !TEX root = ../Lishchuk Diana . Homework 5.1.tex
	
	\chapter{Теоретические обоснования наличия связи между уровнем развития банковской системы и экономическим ростом}
	\section{Положительное влияние банковского сектора на экономический рост}

Существуют множество взглядов на то, каким образом банковский сектор влияет на уровень экономического развития. 

Положительная связь рассмотрена в статье Ross Levine (2004).  Для объяснения наличия данной связи автор представляет четыре функции банковского сектора и объясняет, как каждая из них приводит к росту экономики.
\begin{Enumerate}
	\item \textbf{Сбор и обработка информации о деятельности фирм}.
	
	Без банков каждый инвестор бы сталкивался с огромными издержками, связанными с проверкой деятельности каждой фирмы, её руководителей и финансового положения. Поэтому необходимо наличие коммерческих банков, у которых есть специалисты, которые профессионально занимаются сбором и анализом информации о деятельности предприятий. Это позволяет принимать обоснованное решение о выдаче кредитов фирмам. Если фирма развивается успешно, то финансирование способствует её росту, увеличению производства, что в свою очередь влияет на экономический рост. Но если фирма предпочитает участвовать в рискованных проектах, то с меньшей вероятностью банк ей выдаст кредит из-за риска дефолта заёмщика. Таким образом, банки, собирая информацию о деятельности предприятий, делают возможным кредитование части из них, что позволяет этим фирмам увеличить объемы производства, реализовать инвестиционные проекты.
	\item \textbf{Контроль за деятельностью фирм-заёмщиков}.
	
	Проведение проверки деятельности руководителей фирм, которая должна быть направлена на максимизацию прибыли всей компании, а не отдельных её сотрудников, способствует эффективному распределению денежных средств и побуждает инвесторов вкладывать больше денег в прибыльные и успешные компании. Акционеры могут следить за деятельностью фирмы, чтоб она не вкладывала деньги в рискованные проекты. Когда акционеры уверены в устойчивом состоянии фирмы, то они больше вкладывают, что способствует увеличению производства и развитию инноваций. Однако среди акционеров может возникнуть проблема «безбилетника»: когда один вкладывает собственные деньги в мониторинг фирмы, а другие бесплатно пользуются полученной информацией. В этом случае, каждому акционеру не выгодно финансировать проверку деятельности фирмы, так как его прибыль снижается. Так думают все акционеры, поэтому проверка может быть не осуществлена и менеджеры продолжат работать на собственное благо. В этой ситуации необходимы банки, которые могут решить проблему безбилетника, нанимая специалистов, которые следят за финансовым положением фирм и делают отчёты. Тогда фирмы не будут активно инвестировать в рискованные проекты, так как это приведёт к невозможности взять кредит у банков. В свою очередь, банки, налаживая долгосрочные отношения с фирмами, в будущем будут тратить меньше денег на мониторинг. Тогда у них больше выручка, а, значит, они способны выдать больше кредитов успешным фирмам, что приведёт к увеличению производства и положительному влиянию на экономический рост
	\item \textbf{Диверсификация и управление рисками}.
	
	Вложение средств в различные активы сопровождается рисками, связанными с возможной успешностью или не успешностью инвестиционного проекта. Если агент вложит все свои деньги в один проект, то велик риск потерять свои средства в случае неудачи. Однако банки вкладывают привлеченные средства в различные активы. Диверсификация активов снижает риск вложений. Инвестиции в один проект рискованны, так как есть вероятность, что проект окажется неуспешным и инвестор потеряет все свои деньги. Однако распределение своих средств в несколько проектов разных компаний в разных отраслях экономики ведёт к тому, что вероятность получить прибыль от своих вложений вырастет. Так происходит, потому что если один проект может не принести ожидаемой выручки, то другой может принести немалую прибыль, что компенсирует потери от первого проекта. Тогда фирмам и домохозяйствам выгодно вкладывать свои деньги в банк, который может вложить их с большей отдачей, чем каждый агент по отдельности. Банки имеют больше привлечённых средств, что означает увеличение способности банка выдавать кредиты фирмам, которые могут потратить их на расширение своей деятельности. Таким образом,  в экономике больше хорошо функционирующих фирм, что способствует экономическому росту.
	\item \textbf{Накопление и перемещение денежных средств}.
	
	Банк – это место накопления сбережений от экономических агентов. Посредничество способствует экономии времени и издержек на осуществление сделок.  Если каждый агент, имеющий небольшую сумму денег, желает их выгодно вложить, то он может получить отрицательную выручку за счет больших издержек на проведение данной операции (транзакционные издержки). Однако когда посредники принимают средства от нескольких агентов, они могут все эти счета включить в один контракт, что сократит издержки на привлечение дилера в расчете на одного агента. Такое явление называется экономией от масштаба. Таким образом, у фирм и домохозяйств больше средств, которые можно выгодно вложить, использовать для расширения производства.
\end{Enumerate}
\newcommand{\dl}{\Delta}
\newcommand{\s}{\delta}
\newcommand{\f}{\varphi}

Положительная связь между экономическим ростом и финансовым сектором отражена в модели Столбова. Рассмотрим модель аккумулированного капитала: \[Y=A*K, \tag{1}\label{1}\] где $Y$ - объём выпуска в экономике, $K$ - объём капитала (как физического (основных фондов), так и человеческого), $A$ - уровень технологии. Включаем финансовый рынок, который представляет собой деятельность различных финансовых посредников по преобразованию сбережений в инвестиции, в модель экономического роста: \[\dl K=I-\s*K, \tag{2}\label{2}\] где $\dl K$ - прирост капитала, $I$ - инвестиции, $\s*K$ - объём выбытого капитала; \[S=s*Y, \tag{3}\label{3}\] где $S$ - сбережения, $s$ - норма сбережений;\[I=S-C(S,\f), \tag{4}\label{4}\] где $C(S,\f)$ - издержками финансового посредничества. Например, такие как выплаты  комиссионных вознаграждений финансовым институтам за предоставляемые услуги, различные транзакционные расходы, связанные с инвестиционным процессом и т. п.

Функция издержек финансового посредничества $C(S,\f)$ является функцией двух переменных – объёма сбережений $S$ и эффективности финансового рынка $\f$. По мере роста переменной $S$ функция $C(S,\f)$ возрастает, т.к. чем больше объём сбережений в экономике, тем выше будут совокупные издержки по их трансформации в инвестиции. 

Переменная $\f$ характеризует эффективность функционирования финансового рынка\footnote{В качестве меры эффективности финансового рынка можно использовать довольно широкий круг показателей, например, совокупный объём накладных расходов коммерческих банков или среднее значение транзакционных издержек по оформлению сделок по купле – продаже ценных бумаг на национальных биржах}. По условию модели, это экзогенный параметр. Далее предположим, что чем больше значение $\f$, тем выше степень эффективности финансового рынка. Пусть переменная $\f$ равна величине обратной спрэду процентных ставок по кредитам и депозитам $\frac{1}{r_k-r_d}$, где $r_k, r_d$- ставки по кредитам и депозитам соответственно, выраженные в долях единицы. Тогда функция издержек финансового посредничества возрастает по переменной φ. Это очевидно, т.к. создание эффективного финансового рынка или осуществление мер по его модернизации неизбежно потребуют значительных капиталовложений в его инфраструктуру. Однако в то же время более эффективный финансовый рынок, по определению, должен их минимизировать. Данное противоречие оказывается разрешимо, если функция издержек финансового посредничества принимает следующий вид:Переменная φ характеризует эффективность функционирования финансового рынка . По условию модели, это экзогенный параметр. Далее предположим, что чем больше значение φ, тем выше степень эффективности финансового рынка. Пусть переменная φ равна величине обратной спрэду процентных ставок по кредитам и депозитам $\frac{1}{r_k-r_d}$, где $r_k,r_d$- ставки по кредитам и депозитам соответственно, выраженные в долях единицы. Тогда функция издержек финансового посредничества возрастает по переменной φ. Это очевидно, т.к. создание эффективного финансового рынка или осуществление мер по его модернизации неизбежно потребуют значительных капиталовложений в его инфраструктуру. Однако в то же время более эффективный финансовый рынок, по определению, должен их минимизировать. Данное противоречие оказывается разрешимо, если функция издержек финансового посредничества принимает следующий вид:
\[C(S,\f)=c_0+c_1*\f+\frac{c_2*S}{\f}, \tag{5}\label{5} \] где $c_0,~c_1,~c_2$- положительные коэффициенты. 

Подробнее рассмотрим уравнение \ref{5}. Предположим, что объём сбережений в экономике – единственная переменная, а φ - параметр, то компонент постоянных издержек, равный $FC=c_0+c_1*\f$ , будет возрастать по мере роста эффективности работы финансового рынка, что и объясняет введение значительных капиталовложений в его инфраструктуру. В то же время  предельных издержек будут представлять собой функцию, убывающую по мере улучшения функционирования деятельности финансовых посредников $MC=\frac{c_2}{\f}$. 

Имея уравнения (\ref{1})~--~(\ref{5}), преобразуем уравнение (\ref{2}) следующим образом:
\[\dl K=s*A*K-[c_0+c_1*φ+(δ+\frac{c_2*s*A}{\f})*K]  \tag{6}\label{6} \]

Теперь можно ввести формулу для темпов экономического роста: 
\[g(Y)=\frac{\dl~Y}{Y}=\frac{\dl K}{K}=s*A-\frac{c_0+c_1*\f}{K} +(\s+\frac{c_2*s*A}{\f}) =(\frac{1-c_2}{\f})*s*A-\s-\frac{c_0+c_1*\f}{K} \tag{7}\label{7} \] 

Тогда долгосрочный темп прироста национального дохода равен:
\[g^*(Y)=\lim\limits_{n \to \infty} \left( (1-\frac{c_2}{\f})*s*A-\s-\frac{c_0+c_1*\f}{K}\right) =(1-\frac{c_2}{\f})*s*A-\s  \tag{8}\label{8}\]
\todo{сделай большие скобки}

Таким образом, чем выше уровень эффективности финансовых рынков, тем выше темп экономического роста.

\section{Отрицательное влияние банковского сектора на экономический рост}

Отрицательная связь рассмотрена в своей статье Cecchetti и Kharroubi (2004). Авторы показывают, что когда банки предъявляют высокие требования к заемщикам, часть высокоприбыльных и рискованных проектов не реализуется, что, потенциально снижает темпы роста экономики. 

Авторы рассматривают два типа агентов – фирмы и банки, которые действуют в течение одного периода и имеют первоначальный запас средств $e$ и $f$, соответственно. В начале каждого периода фирмы принимают решение о том, какой проект реализовать и сколько средств нужно занять $d$, а банки – об объеме выданных кредитов и валовой ставке $r$ по кредитам. 

Существуют  две группы фирм: одна половина фирм предпочитает вкладывать деньги в высокоприбыльные проекты,  другая – в низкоприбыльные проекты. Обозначим эти проекты $a$ и $b$ соответственно, валовая доходность которых равна $R_a$ и $R_b$, где $R_a > R_b >1$. Высокоприбыльные проекты более рискованны и менее желаемы для банков. Фирмы, предпочитающие высокоприбыльные проекты, рискуют потерять вложенные деньги, что затрудняет возврат долга банкам. Таким образом, банки с высокой вероятностью могут потерять свои деньги из-за рискованных фирм. Кроме того предполагается, что издержки банков, связанные с кредитованием фирм, увеличиваются за счет издержек на возврат средств после дефолта заёмщика. Следовательно, банки предпочитают вторую группу фирм, вкладывающих полученную сумму денег в низкоприбыльные, но более надёжные проекты, так как увеличивается вероятность возврата средств. 

Банки стимулируют фирмы осуществлять низкорискованные проекты, требуя у них залог, равный доле $p_i$ от общей суммы имеющихся денежных средств, где $i = a,b$. Чем больше величина ставки залога, тем меньше у фирм стимулов для участия в рискованной деятельности, так как они много потеряют в случае дефолта. Тогда получаем, что чем меньше объём залога требуют оставить банки, тем выше стимулы у фирм вложиться в рискованный высокоприбыльный проект: $p_a < p_b$, т.к. $R_a > R_b$. 

Также в модели предполагается, что в случае дефолта фирма обязана вернуть часть одолженной суммы, то есть долю кредита в размере $P$. Таким образом, фирма, вкладывая в проект сумму ($e+d$) получает выручку в размере $R*(e+d)$, при этом отдавая залог за кредит в размере $p*(e+d)$  и возвращая основную сумму долга и проценты rd, в случае если её вложения оказались успешны, и Prd, в случае дефолта заёмщика:
\renewcommand{\theequation}{\arabic{equation}}
\begin{equation}
\left \{
\begin{aligned}
&(e+d)*(R_i-p_i )-r*d,   \quad \text{если проект успешен} \\
&(e+d)*(R_i-p_i )-P*r*d, \quad \text{если проект не успешен}
\end{aligned}
\right.
\end{equation}

Так как в модели предполагается, что банк имеет возможность самостоятельно устанавливать размер залога, то можно сделать вывод о том, что он выберет такую величину, чтобы фирме было невыгодно вкладывать средства в рискованный проект. Как говорилось выше, банки предпочитают выдавать кредиты фирмам, участвующим в менее рискованных проектах. Таким образом, высокая ставка по залогу приводит к тому, что часть высокоприбыльных проектов не осуществляются, фирмы теряют часть своей возможной выручки, которая могла бы пойти на создание более эффективного способа производства, снижающего издержки фирм, что, в свою очередь, сдерживает рост фирм и всей экономики.