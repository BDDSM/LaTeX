\chapter*{Введение}

\addcontentsline{toc}{chapter}{Введение}


Облигация является одним из~важнейших финансовых инструментов. Она позволяет государственным органам и компаниям привлекать необходимые средства для~своей деятельности, зачастую, с~более низкими процентными платежами и на~более длительный срок, чем через кредиты. Кроме того, облигации рассчитаны на~широкий круг инвесторов, что делает возможности привлечения средств через этот инструмент практически неограниченными.

Существует три вида облигаций: 
\begin{enumerate}
	\item государственные;
	\item муниципальные;
	\item корпоративные.
\end{enumerate}
Корпоративные облигации несут на~себе так называемый риск~дефолта (или кредитный~риск)~– риск того, что эмитент облигации  откажется или будет не в~состоянии осуществлять предусмотренные купонные выплаты или выплатить номинал, когда придет срок погашения ценной бумаги. Конечно, государство или местные органы власти также могут отказаться расплачиваться по своим долгам (как, например, сделало правительство России в~1998~году), но в~науке государственные облигации, как и муниципальные, воспринимаются как свободные от~риска дефолта, поскольку считается, что государство всегда может увеличить налоги для~того, чтобы рассчитаться по~своим облигациям.

Для оценки риска, связанного с~корпоративной облигацией, специальные кредитные агентства присваивают эмитентам и выпускам облигаций кредитные рейтинги. Кредитный рейтинг выражают мнение агентства относительно способности эмитента своевременно и в~полном объеме выполнить свои финансовые обязательства. Он отражает кредитное качество долгового обязательства и относительную вероятность дефолта по~нему.

Кредитные рейтинги облигации и её эмитента непосредственно влияют на~желание агентов приобретать данную ценную бумагу, поскольку они являются рискофобами . Соответственно, чем выше риски, связанные с облигацией, тем меньше желающих приобрести её, а~значит, тем сложнее эмитенту получить столько необходимые для~его деятельности средства.

Поэтому особенно важным становится ответ на~вопрос~: что и как влияет на риски, связанные с корпоративными облигациями. Поиску ответа на~этот вопрос и посвящена данная работа.
Целью работы является анализ факторов рисков на рынке корпоративных облигаций в~России. Для этого ставятся следующие задачи~:
\begin{enumerate}
	\item обзор литературы;
	\item сбор и анализ данных;
	\item характеристика рисков, связанных с российскими корпоративными облигациями;
	\item выявление факторов, влияющих на риски, связанные с корпоративными облигациями российских компаний;
	\item предоставление практических рекомендаций по снижению этих рисков.
\end{enumerate}
