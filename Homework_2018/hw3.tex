%!TEX TS-program = xelatex
\documentclass[12pt, a4paper, oneside]{article}

\usepackage{amsmath,amsfonts,amssymb,amsthm,mathtools}  % пакеты для математики

\usepackage[utf8]{inputenc} % задание utf8 кодировки исходного tex файла
\usepackage[british,russian]{babel} % выбор языка для документа

\usepackage{fontspec}         % пакет для подгрузки шрифтов
\setmainfont{Helvetica}   % задаёт основной шрифт документа

% why do we need \newfontfamily:
% http://tex.stackexchange.com/questions/91507/
\newfontfamily{\cyrillicfonttt}{Helvetica}
\newfontfamily{\cyrillicfont}{Helvetica}
\newfontfamily{\cyrillicfontsf}{Helvetica}

\usepackage{unicode-math}     % пакет для установки математического шрифта
\setmathfont{Neo Euler}      % шрифт для математики
% \setmathfont[math-style=ISO]{Asana Math}
% Можно делать смену начертания с помощью разных стилей

% Конкретный символ из конкретного шрифта
% \setmathfont[range=\int]{Neo Euler}

%%%%%%%%%% Работа с картинками %%%%%%%%%
\usepackage{graphicx}                  % Для вставки рисунков
\usepackage{graphics}
\graphicspath{{images/}{pictures/}}    % можно указать папки с картинками
\usepackage{wrapfig}                   % Обтекание рисунков и таблиц текстом

%%%%%%%%%%%%%%%%%%%%%%%% Графики и рисование %%%%%%%%%%%%%%%%%%%%%%%%%%%%%%%%%
\usepackage{tikz, pgfplots}  % язык для рисования графики из latex'a

%%%%%%%%%% Гиперссылки %%%%%%%%%%
\usepackage{xcolor}              % разные цвета

\usepackage{hyperref}
\hypersetup{
	unicode=true,           % позволяет использовать юникодные символы
	colorlinks=true,       	% true - цветные ссылки, false - ссылки в рамках
	urlcolor=blue,          % цвет ссылки на url
	linkcolor=red,          % внутренние ссылки
	citecolor=green,        % на библиографию
	pdfnewwindow=true,      % при щелчке в pdf на ссылку откроется новый pdf
	breaklinks              % если ссылка не умещается в одну строку, разбивать ли ее на две части?
}


\usepackage{todonotes} % для вставки в документ заметок о том, что осталось сделать
% \todo{Здесь надо коэффициенты исправить}
% \missingfigure{Здесь будет Последний день Помпеи}
% \listoftodos --- печатает все поставленные \todo'шки

\usepackage{enumitem} % дополнительные плюшки для списков
%  например \begin{enumerate}[resume] позволяет продолжить нумерацию в новом списке

\usepackage[paper=a4paper, top=20mm, bottom=15mm,left=20mm,right=15mm]{geometry}
\usepackage{indentfirst}       % установка отступа в первом абзаце главы

\usepackage{setspace}
\setstretch{1.15}  % Межстрочный интервал
\setlength{\parskip}{4mm}   % Расстояние между абзацами
% Разные длины в латехе https://en.wikibooks.org/wiki/LaTeX/Lengths


\usepackage{xcolor} % Enabling mixing colors and color's call by 'svgnames'

\definecolor{MyColor1}{rgb}{0.2,0.4,0.6} %mix personal color
\newcommand{\textb}{\color{Black} \usefont{OT1}{lmss}{m}{n}}
\newcommand{\blue}{\color{MyColor1} \usefont{OT1}{lmss}{m}{n}}
\newcommand{\blueb}{\color{MyColor1} \usefont{OT1}{lmss}{b}{n}}
\newcommand{\red}{\color{LightCoral} \usefont{OT1}{lmss}{m}{n}}
\newcommand{\green}{\color{Turquoise} \usefont{OT1}{lmss}{m}{n}}

\usepackage{titlesec}
\usepackage{sectsty}
%%%%%%%%%%%%%%%%%%%%%%%%
%set section/subsections HEADINGS font and color
\sectionfont{\color{MyColor1}}  % sets colour of sections
\subsectionfont{\color{MyColor1}}  % sets colour of sections

%set section enumerator to arabic number (see footnotes markings alternatives)
\renewcommand\thesection{\arabic{section}.} %define sections numbering
\renewcommand\thesubsection{\thesection\arabic{subsection}} %subsec.num.

%define new section style
\newcommand{\mysection}{
	\titleformat{\section} [runin] {\usefont{OT1}{lmss}{b}{n}\color{MyColor1}} 
	{\thesection} {3pt} {} } 


%	CAPTIONS
\usepackage{caption}
\usepackage{subcaption}
%%%%%%%%%%%%%%%%%%%%%%%%
\captionsetup[figure]{labelfont={color=Turquoise}}

\usepackage[normalem]{ulem}  % для зачекивания текста

\pagestyle{empty}

\begin{document}

\section*{Задание 3 (20 + 15 баллов)  }

Не забывай, где находится  \href{https://fulyankin.github.io/LaTeX/}{страничку курса} с кучей шпаргалок!

\textbf{Внимание!}  Вы можете получить то количество баллов, которое вам хочется получить. (Это правда, вы очень большой молодец, когда не ленитесь. Более того, вы ещё и очень красивы. Также, в отличие от вас, у автора задач очень страшная внешность. А ещё он картавый, безграмотный и должен страдать.)

\textbf{Обратите внимание, что первое упражнение обязательное, а второе бонусное.} Когда вы сделали ровно столько задач, сколько хотите, то вы должны:

\begin{enumerate}
\item Убедиться, что сейчас не 11 часов утра 14 марта и дедлайн по домашке ещё не прошёл.
\item Проверить точно ли файл без ошибок компилируется на вашем компьютере.
\item Оформляйте каждое упражнение в отдельном файле (всего три файла).
\item Удалите все промежуточные файлы. В папке должны остаться только .tex, .pdf, картинки. Если вы использовали нестандартный шрифт, приложите файл с ним к архиву.
\item Положить архив в	свой	Dropbox,	Github,	yandex-disk	или
другой	репозиторий.
\item Заполнить	\href{https://docs.google.com/forms/d/e/1FAIpQLSe11kxKVfv07iCL1E9yNX7ll9swKImiVwRr1H70lslGzInRSg/viewform}{уютную гугл-форму.} Ради всего святого называйте свои папки в формате: номер дз Фамилия Имя. Например: 2 Петров Пётр.
\item Не стесняйтесь абсолютно в любое время дня и ночи просить о помощи, если она вам действительно необходима! \textbf{Также не забывайте про то, что любое творчество поощряется. А ещё, что в преамбуле новые команды и окружения можно прописывать с помощью Tikz - прибамбасов! }
\item Если ты девушка, то с приближающимся 8 марта тебя! Если ты парень, то не забудь накупить цветов и шоколада. Иначе в следующем году ты не получишь пену для бритья и носки!
\item Команда курса подготовила для девушек сюрприз на 8 марта. Мы дарим вам самое дорогое, что может быть в этом бренном мире --- время.  Если вы девушка, то для вас дедлайн передвигается с 11:00 на 11:30. С праздником вас, наши любимые дамы!
\end{enumerate}


\subsection*{[5]   Упражнение 1 (Пожалуй, самый сложный пакет в вашей жизни ) }

Суть этого упражнения сводится к тому, что вы должны заставить работать пакет minted.  Этот пакет вы в будущем будете использовать для оформления кода.  Напоминаю, что этот пакет написан на Python. Это означает, что для использования minted должен быть установлен Python. \textbf{Следуя инструкции ниже, установите minted.} 

\textbf{Windows:}

\begin{enumerate}
\item Устанавливаем на компьютер Python. Лучше всего поставить дистрибутив, который называется \href{https://docs.continuum.io/anaconda/install}{Anaconda.} Этот дистрибутив включает в себя все основные пакеты, которые необходимы для работы с питоном. 

\item Открываем консоль. Для этого жмём \texttt{win+R}, вводим в открывшемся окне \texttt{cmd}, жмём \texttt{enter}.  Открывается командная строка. 

\item Прописываем в командной строке \texttt{pip install Pygments}

\item Команда выше установила на наш компьютер питоновский пакет, который будет раскрашивать код в \LaTeX{}. Теперь нужно настроить texstudio. Заходим в настройки и там прописываем в графе  XeLatex: \newline  \texttt{Xelatex -shell-escape -synctex=1 -interaction=nonstopmode \%.tex`}

\item Эта команда подключает к теху внешние пакеты. В нашем случае это Pygments. 
\end{enumerate} 


\textbf{Linux (Ubuntu 16):}

\begin{enumerate}
\item Если честно, то в Anaconda много бесполезного хлама. И лучше приручать змей вручную. Но не на Windows. Жмём \texttt{ctrl+alt+T}, открывается терминал. 
\item Убеждаемся, что установлен Python, вбивая в терминале \texttt{python --version}, а он установлен, потому что половина системы на нём написана.
\item Убеждаемся, что установлен pip, вбивая  \texttt{pip --version}. Если он не установлен, то ставим его!  \texttt{sudo apt-get install python-pip}.
\item Устанавливаем наш пакет  \texttt{sudo pip install Pygments}
\item Заходим в настройки texmaker и там прописываем в графе  XeLatex:  \newline \texttt{Xelatex -shell-escape -synctex=1 -interaction=nonstopmode \%.tex`}
\end{enumerate} 

\textbf{Mac:}

\begin{enumerate}
\item Оказывается, у вас на macOS уже стоит python. Еси открыть терминал, то можно убедиться в этом, прописав \texttt{python --version}
\item Убеждаемся, что установлен pip, вбивая  \texttt{pip --version}. Если он не установлен, то ставим его!  
\item Устанавливаем наш пакет  \texttt{sudo pip install Pygments}
\item  Готово! Теперь, если вы спросите \texttt{which pygmentize}, то  ответ должен быть такой  \texttt{pygmentize is /usr/local/bin/pygmentize}
\item Теперь можно запускать техмейкер/техстудио/техпад, подключать minted и, если вы не забыли в настройках  в графе  XeLatex:  подписать  \texttt{Xelatex -shell-escape -synctex=1 -interaction=nonstopmode \%.tex`}, то всё должно работать.
\end{enumerate} 


После всех этих действий вы должны почувствовать себя супермега программистом. Дело осталось за малым. Создаём теховский документ, подключаем пакет minted и используем окружение minted. Вы ещё не забыли, что задание состояло в том, что нужно оформить какой-нибудь кусочек своего кода с помощью minted? Именно за оформление любого кода вы и получите свои честно заработанные баллы. 



\subsection*{[15]   Упражнение 2 (Тот, чьё имя никто не знает)}

Поздравляю! Вы тот, кто рассылает письма из Хогвартса! И вы столкнулись с проблемой. Министерство \sout{Образования РФ}  Магии недавно приняло совершенно дурацкий законопроект, который задаёт единый образец для оформления всех идущих из Хогвартса писем. Как же хорошо, что в мире волшебников есть \LaTeX{}, недоступный для маглов. Ведь он поможет один раз и навсегда написать такой шаблон. 

\textbf{Требования к письму из школы:}
\begin{enumerate}
\item Текст письма должен быть написан на листе формата A4
\item Текст печатается с единичным интервалом между строками шрифтом Arial 14 размера
\item Поля страницы должны быть отрегулированны следующим образом: отступы сверху и снизу листа по 10мм, слева и справа по 35мм
\item Никаких красных строк в начале абзаца, расстояние между абзацами 4мм
\item  Наверху, перед текстом письма, ровно по центру должен быть расположен герб школы чародейства и волшебства. Высота картинки должна быть ровно 6 см.
\item  Ровно через 1 см от герба расположена информация о том кому именно предназначено это письмо. Например, "Мистеру Поттеру". Эта информация набрана 12 шрифтом.
\item  Ровно через 2.5 см от информации о получателе начинается текст письма. Текст может быть любым. Главное, чтобы в нём было отражено, что получатель приглашается для обучения в Хогвартс.
\item  Информация, которая идёт после текста письма должна быть всегда прибита к низу страницы вне зависимости от того насколько длинным получилось письмо.
\item Подпись профессора МакГонагалл может быть выполнена любым шрифтом, имитирующим человеческую подпись. Также она может быть прорисована в TiKz подобно xkcd стилю.
\item  В самом низу есть гиперссылка на сайт Школы Чародейства и Волшебства.
\item Страница с письмом не нумеруется.
\end{enumerate}


\textbf{Требования к приложению:}
\begin{enumerate}[resume]
\item  В приложении А располагаются список необходимых книг и предметов.  Приложения нумеруются русскими буквами. Перед номером приложения идёт слово "Приложение".
\item  Каждый новый предмет идёт после изображения волшебной палочки.
\item  В приложении Б располагается список предметов, изучаемых в течение первого учебного года.
\item  На каждой странице расположены колонтитулы, которые необходимо оформить также как в образце.
\end{enumerate}

При оформлении письма оставьте в преамбуле только те пакеты, которые реально используются в документе.

\begin{itemize}
\item[$(10)$] Вы оформили письмо в соответствии с требованиями.

\item[$(5)$] Вы отправили письмо другу, который не ходит на факультатив. В качестве подтверждения факта отправки прикреплён скрин с реакцией друга на письмо.
\end{itemize}


\subsection*{[15]  Упражнение 3 (Резюме)}

\textbf{Это упражнение не является обязательным.}  Когда человек хочет найти работу, он пишет резюме. Скорее всего, вы тоже рано или поздно будете искать работу. Пришло время написать себе резюме с помощью \LaTeX{}!

Вы можете сделать это самостоятельно или воспользоваться одним из шаблонов. Вы можете сделать это как на русском, так и на английском языке. Не забудьте указать  \LaTeX{} в числе программ, которыми вы умеете пользоваться!

\begin{itemize}
\item[$(10)$] Вы сделали адекватное резюме.
\item[$(5)$] Увидев ваше резюме, я захотел, чтобы вы сделали для меня кое-какую грязную работку.
\end{itemize}


\subsection*{ [Бесценно]  Упражнение 4}

Начни уже писать свой НИР! Желательно в  \LaTeX.  Иначе придётся начать это делать через неделю. 

\end{document}
