%!TEX TS-program = xelatex
\documentclass[12pt, a4paper]{article}

% Этот шаблон документа разработан в 2014 году
% Данилом Фёдоровых (danil@fedorovykh.ru) 
% для использования в курсе 
% <<Документы и презентации в \LaTeX>>, записанном НИУ ВШЭ
% для Coursera.org: http://coursera.org/course/latex .
% Исходная версия шаблона --- 
% https://www.writelatex.com/coursera/latex/5.3

% В этом документе преамбула

%%% Работа с русским языком
\usepackage{cmap}					% поиск в PDF
\usepackage{mathtext} 				% русские буквы в формулах
\usepackage[T2A]{fontenc}			% кодировка
\usepackage[utf8]{inputenc}			% кодировка исходного текста
\usepackage[english,russian]{babel}	% локализация и переносы
\usepackage{indentfirst}
\frenchspacing

\renewcommand{\epsilon}{\ensuremath{\varepsilon}}
\renewcommand{\phi}{\ensuremath{\varphi}}
\renewcommand{\kappa}{\ensuremath{\varkappa}}
\renewcommand{\le}{\ensuremath{\leqslant}}
\renewcommand{\leq}{\ensuremath{\leqslant}}
\renewcommand{\ge}{\ensuremath{\geqslant}}
\renewcommand{\geq}{\ensuremath{\geqslant}}
\renewcommand{\emptyset}{\varnothing}

%%% Дополнительная работа с математикой
\usepackage{amsmath,amsfonts,amssymb,amsthm,mathtools} % AMS
\usepackage{icomma} % "Умная" запятая: $0,2$ --- число, $0, 2$ --- перечисление

%% Номера формул
%\mathtoolsset{showonlyrefs=true} % Показывать номера только у тех формул, на которые есть \eqref{} в тексте.
%\usepackage{leqno} % Нумереация формул слева

%% Свои команды
\DeclareMathOperator{\sgn}{\mathop{sgn}}

%% Перенос знаков в формулах (по Львовскому)
\newcommand*{\hm}[1]{#1\nobreak\discretionary{}
{\hbox{$\mathsurround=0pt #1$}}{}}

%%% Работа с картинками
\usepackage{graphicx}  % Для вставки рисунков
\graphicspath{{images/}{images2/}}  % папки с картинками
\setlength\fboxsep{3pt} % Отступ рамки \fbox{} от рисунка
\setlength\fboxrule{1pt} % Толщина линий рамки \fbox{}
\usepackage{wrapfig} % Обтекание рисунков текстом

%%% Работа с таблицами
\usepackage{array,tabularx,tabulary,booktabs} % Дополнительная работа с таблицами
\usepackage{longtable}  % Длинные таблицы
\usepackage{multirow} % Слияние строк в таблице

%%% Теоремы
\theoremstyle{plain} % Это стиль по умолчанию, его можно не переопределять.
\newtheorem{theorem}{Теорема}[section]
\newtheorem{proposition}[theorem]{Утверждение}
 
\theoremstyle{definition} % "Определение"
\newtheorem{corollary}{Следствие}[theorem]
\newtheorem{problem}{Задача}[section]
 
\theoremstyle{remark} % "Примечание"
\newtheorem*{nonum}{Решение}

%%% Программирование
\usepackage{etoolbox} % логические операторы

%%% Страница
\usepackage{extsizes} % Возможность сделать 14-й шрифт
\usepackage{geometry} % Простой способ задавать поля
	\geometry{top=25mm}
	\geometry{bottom=35mm}
	\geometry{left=35mm}
	\geometry{right=20mm}
 %
%\usepackage{fancyhdr} % Колонтитулы
% 	\pagestyle{fancy}
 	%\renewcommand{\headrulewidth}{0pt}  % Толщина линейки, отчеркивающей верхний колонтитул
% 	\lfoot{Нижний левый}
% 	\rfoot{Нижний правый}
% 	\rhead{Верхний правый}
% 	\chead{Верхний в центре}
% 	\lhead{Верхний левый}
%	\cfoot{Нижний в центре} % По умолчанию здесь номер страницы

\usepackage{setspace} % Интерлиньяж
%\onehalfspacing % Интерлиньяж 1.5
%\doublespacing % Интерлиньяж 2
%\singlespacing % Интерлиньяж 1

\usepackage{lastpage} % Узнать, сколько всего страниц в документе.

\usepackage{soul} % Модификаторы начертания

\usepackage{hyperref}
\usepackage[usenames,dvipsnames,svgnames,table,rgb]{xcolor}
\hypersetup{				% Гиперссылки
    unicode=true,           % русские буквы в раздела PDF
    pdftitle={Заголовок},   % Заголовок
    pdfauthor={Автор},      % Автор
    pdfsubject={Тема},      % Тема
    pdfcreator={Создатель}, % Создатель
    pdfproducer={Производитель}, % Производитель
    pdfkeywords={keyword1} {key2} {key3}, % Ключевые слова
    colorlinks=true,       	% false: ссылки в рамках; true: цветные ссылки
    linkcolor=red,          % внутренние ссылки
    citecolor=black,        % на библиографию
    filecolor=magenta,      % на файлы
    urlcolor=cyan           % на URL
}

\usepackage{csquotes} % Еще инструменты для ссылок

%\usepackage[style=authoryear,maxcitenames=2,backend=biber,sorting=nty]{biblatex}

\usepackage{multicol} % Несколько колонок

\usepackage{tikz} % Работа с графикой
\usepackage{pgfplots}
\usepackage{pgfplotstable}




% Специальный пакет для оформления задач! 
\newtheorem{problem}{Задача}

\usepackage{answers}
\Newassociation{sol}{solution}{solution_file}
% sol --- имя окружения внутри задач
% solution --- имя окружения внутри solution_file
% solution_file --- имя файла в который будет идти запись решений


\begin{document}

% Открываем файл, куда будут записываться решения. 
\Opensolutionfile{solution_file}[all_solutions]


\begin{problem}
В России 1112 городов (наверное). Ярополк посетил половину. Мирослав тоже. Эти половины не совпадают. Оба они побывали в четверти этих городов. Сколько городов не посетил ни один из них?
\begin{sol}

\end{sol}
\end{problem}



\begin{problem}
Попечительский совет выделил ученикам 10 класса денежные средства на сбор статистики. На эти деньги ученики должны были купить фломастеры, бумагу и т.п. Ученики собрали информацию следующего характера. В школе 45 учеников, в том числе 25 мальчиков. На хорошо и отлично учатся 30 учеников, в том числе 16 мальчиков. Спортом заняты 28 учеников, среди которых 18 мальчиков и 17 учеников, учащихся на отлично и хорошо. 15 мальчиков учатся на отлично и хорошо и занимаются спортом.
Сторож Афанасий, совершавший вечерний обход помещений, заподозрил что-то неладное, заметив за одной из батарей в школьном коридоре спрятанную пустую бутылку с надписью "Хортица". Об этом он сообщил директрисе. После тщательной проверки собранной статистики ученики 10 класса получили выговор. Что именно нашла директриса в отчёте 10-классников? 
\begin{sol}

\end{sol}
\end{problem}



\begin{problem}
Сколько существует способов расселить 5 туристов по 3 домикам так, чтобы ни один домик не оказался пустым? Все туристы и домики различны. Способы расселения отличающиеся только перестановкой туристов, заселённых в один домик, считаются одинаковыми.
\begin{sol}

\end{sol}
\end{problem}



\begin{problem}
В институте работают 100 сотрудников. Английский знают 60 из них. Немецкий не знают 55 из них. Французский знают 25 из них. 25 не знают ни французского, ни английского. Английский и немецкий знают 15. Французский и немецкий знают 15. Все три языка знают 5. 
\begin{itemize}
\item Сколько человек знают немецкий?
\item Сколько человек знают английсий и французский? 
\item Сколько человек знают хотя бы один язык?
\item Сколько человек знают немецкий и французский, но не знают английского?
\item Сколько человек знают французский, но не знают ни английского ни немецкого?
\end{itemize}
\begin{sol}

\end{sol}
\end{problem}



\begin{problem}
Эконом играет в киллера. Всего играет $n$ человек. Каждый получает конверт со своей жертвой. Сколько существует способов раздать конверты так, чтобы ни один человек не получил в качестве своей жертвы себя? 
\begin{sol}

\end{sol}
\end{problem}



\begin{problem}
Три художника пытаются воссоздать картину Ван Гога. Художники встретились, когда у каждого было готово не менее $^2/_3$ картины. Может ли быть, что у каждых двоих общая часть нарисованного составляет не более $^1/_4$ исходной картины?
\begin{sol}

\end{sol}
\end{problem}



\begin{problem}
Переплётчик должен переплести 12 различных книг в красный, зелёный и коричневый переплёты. Сколькими способами он может это сделать, если в каждый цвет должна быть переплетена хотя бы одна книга? (Подсказка: смотри упражнение 3!)
\begin{sol}

\end{sol}
\end{problem}



\begin{problem}
Сколько целых чисел от 0 до 999 не делятся ни на 5 ни на 7? А сколько чисел от 0 до 999 не делятся ни на 2, ни на 3, ни на 5, ни на 7?
\begin{sol}

\end{sol}
\end{problem}



\begin{problem}
Сколько целых чисел от 100 до 1000 не делятся ни на одно из чисел 6, 9 и 15? Делятся ровно на одно из этих чисел? Делятся ровно на два из них? 
\begin{sol}

\end{sol}
\end{problem}





% Закрываем файл, куда мы записывали решения и вставляем его в конце списка задач. 
\Closesolutionfile{solution_file}

% Вставляем решения. Можно их не вставлять или настроить пакет так, чтобы они шли непосредственно после каждой задачи.
% \begin{solution}{1}
\end{solution}
\begin{solution}{3}
\end{solution}
\begin{solution}{4}
\end{solution}



\end{document}
