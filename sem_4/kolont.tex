\documentclass[12pt, a4paper]{article}  

\usepackage{etex} % расширение классического tex в частности позволяет подгружать гораздо больше пакетов, чем мы и займёмся далее

%%%%%%%%%% Математика %%%%%%%%%%
\usepackage{amsmath,amsfonts,amssymb,amsthm,mathtools} 
%\mathtoolsset{showonlyrefs=true}  % Показывать номера только у тех формул, на которые есть \eqref{} в тексте.
%\usepackage{leqno} % Нумерация формул слева


%%%%%%%%%%%%%%%%%%%%%%%% Шрифты %%%%%%%%%%%%%%%%%%%%%%%%%%%%%%%%%
\usepackage{fontspec}         % пакет для подгрузки шрифтов
\setmainfont{HelveticaNeueCyr}   % задаёт основной шрифт документа

% why do we need \newfontfamily:
% http://tex.stackexchange.com/questions/91507/
\newfontfamily{\cyrillicfonttt}{HelveticaNeueCyr}
\newfontfamily{\cyrillicfont}{HelveticaNeueCyr}
\newfontfamily{\cyrillicfontsf}{HelveticaNeueCyr}
% Иногда тех не видит структуры шрифтов. Эти трое бравых парней спасают ситуацию и доопределяют те куски, которые Тех не увидел.

\usepackage{unicode-math}     % пакет для установки математического шрифта
\setmathfont{Asana Math}      % шрифт для математики

\usepackage{polyglossia}      % Пакет, который позволяет подгружать русские буквы
\setdefaultlanguage{russian}  % Основной язык документа
\setotherlanguage{english}    % Второстепенный язык документа



%%%%%%%%%% Работа с картинками %%%%%%%%%
\usepackage{graphicx}                  % Для вставки рисунков
\usepackage{graphics} 
\graphicspath{{images/}{pictures/}}    % можно указать папки с картинками
\usepackage{wrapfig}                   % Обтекание рисунков и таблиц текстом
\usepackage{subfigure}                 % для создания нескольких рисунков внутри одного


%%%%%%%%%% Работа с таблицами %%%%%%%%%%
\usepackage{tabularx}            % новые типы колонок
\usepackage{tabulary}            % и ещё новые типы колонок
\usepackage{array}               % Дополнительная работа с таблицами
\usepackage{longtable}           % Длинные таблицы
\usepackage{multirow}            % Слияние строк в таблице
\usepackage{float}               % возможность позиционировать объекты в нужном месте 
\usepackage{booktabs}            % таблицы как в книгах!  
\renewcommand{\arraystretch}{1.3} % больше расстояние между строками


%%%%%%%%%% Гиперссылки %%%%%%%%%%
\usepackage{xcolor}              % разные цвета

% Два способа включить в пакете какие-то опции:
%\usepackage[опции]{пакет}
%\usepackage[unicode,colorlinks=true,hyperindex,breaklinks]{hyperref}

\usepackage{hyperref}
\hypersetup{				
    unicode=true,           % позволяет использовать юникодные символы
    colorlinks=true,       	% true - цветные ссылки, false - ссылки в рамках
    urlcolor=blue,          % цвет ссылки на url
    linkcolor=red,          % внутренние ссылки
    citecolor=green,        % на библиографию
	pdfnewwindow=true       % при щелчке в pdf на ссылку откроется новый pdf
	hyperindex=true         % сделать ли ссылку кликабельной?
	breaklinks=true         % если ссылка не умещается в одну строку, разбивать    
	                        % ли ее на две части?
}

\usepackage{csquotes}            % Еще инструменты для ссылок

%%%%%%%%%% Програмный код %%%%%%%%%%
\usepackage{minted}
% Включает подцветку комманд в программах!
% Нужно, чтобы на компе стоял питон, надо поставить пакет Pygments, в котором он сделан через pip или cmd ( нужно ввести easy_install Pygments ). 
% После нужно зайти в настройки texmaker и там прописать в PdfLatex pdflatex -shell-escape -synctex=1 -interaction=nonstopmode %.tex
% Документация по пакету хорошая, сам читал, погуглите!


%%%%%%%%%% Другие приятные пакеты %%%%%%%%%
\usepackage{multicol}       % несколько колонок
\usepackage{verbatim}       % для многострочных комментариев
\usepackage{mdframed}       % Этот пакет позволяет рисовать красивые рамки!
\usepackage{makeidx}        % для создания предметных указателей

\usepackage{enumitem} % дополнительные плюшки для списков
%  например \begin{enumerate}[resume] позволяет продолжить нумерацию в новом списке

\usepackage{todonotes} % для вставки в документ заметок о том, что осталось сделать
% \todo{Здесь надо коэффициенты исправить}
% \missingfigure{Здесь будет Последний день Помпеи}
% \listoftodos --- печатает все поставленные \todo'шки



%%%%%%%%%% Оформление %%%%%%%%%%
%ТОТ САМЫЙ ПАКЕТ - ЛЕГЕНДА!

% Сделать норм порядок пакетов!
% https://en.wikibooks.org/wiki/LaTeX/Lengths   разные длины которые есть в LaTeX

\usepackage{indentfirst}       % установка отступа в первом абзаце главы!!!

\usepackage{setspace}  
\setstretch{1.255}  % Межстрочный интервал
\setlength{\parindent}{1.5em} % Красная строка.
\usepackage{extsizes} % Возможность сделать 14-й шрифт

% размер листа бумаги
\usepackage[paper=a4paper,top=15mm, bottom=15mm,left=35mm,right=10mm,includefoot]{geometry}

\usepackage{fancyhdr} % Колонтитулы
\pagestyle{fancy}

\renewcommand{\headrulewidth}{0.4pt}  % Толщина линий, отчеркивающих верхний
\renewcommand{\footrulewidth}{0.4pt}  % и нижний колонтитулы
	\lfoot{Нижний левый}
	\rfoot{Нижний правый}
	\rhead{Верхний правый}
 	\chead{Верхний в центре}
	\lhead{Верхний левый}
	\cfoot{ }         % Тут по умолчанию печатается номер страницы 
	%\lhead{\thepage} % Новая нумерация



\flushbottom                            % Эта команда заставляет LaTeX чуть растягивать строки, чтобы получить идеально прямоугольную страницу
\righthyphenmin=2                       % Разрешение переноса двух и более символов
\widowpenalty=300                     % Небольшое наказание за вдовствующую строку (одна строка абзаца на этой странице, остальное --- на следующей)
\clubpenalty=3000                     % Приличное наказание за сиротствующую строку (омерзительно висящая одинокая строка в начале страницы)
\tolerance=1000     % Ещё какое-то наказание.






\begin{document}

\thispagestyle{empty}

\begin{center}
\small Российская академия народного хозяйства

 и государственной службы при Президенте РФ

 \rule{\linewidth}{0.5pt}

\vfill\normalsize НАУЧНО-ИССЛЕДОВАТЕЛЬСКАЯ РАБОТА НА ТЕМУ:

\vfill\normalsize


\Large\textsc{ \textbf{Написание сложных формул с закрытыми глазами и фальсификация расчётов так, чтобы всё было ок и тебя не спалили
}}\\

\vfill
\normalsize Москва\\ 2016

\end{center}

\newpage


%автоматическое оглавление
\tableofcontents

% Задать вопрос про то почему оно красное и как от этого избавиться.

\section{Первая часть}
\subsection{Первый кусок текста в первой части}

Бла бла бла бла бла бла бла бла бла бла. Бла бла бла бла бла бла бла бла бла бла. Бла бла бла бла бла бла бла бла бла бла. Бла бла бла бла бла бла бла бла бла бла. Бла бла бла бла бла бла бла бла бла бла. Бла бла бла бла бла бла бла бла бла бла.Бла бла бла бла бла бла бла бла бла бла. Бла бла бла бла бла бла бла бла бла бла. Бла бла бла бла бла бла бла бла бла бла. Бла бла бла бла бла бла бла бла бла бла. Бла бла бла бла бла бла бла бла бла бла. Бла бла бла бла бла бла бла бла бла бла.Бла бла бла бла бла бла бла бла бла бла. Бла бла бла бла бла бла бла бла бла бла.

\subsection{Второй кусок текста в первой части}
Бла бла бла бла бла бла бла бла бла бла. Бла бла бла бла бла бла бла бла бла бла. Бла бла бла бла бла бла бла бла бла бла. Бла бла бла бла бла бла бла бла бла бла. Бла бла бла бла бла бла бла бла бла бла. Бла бла бла бла бла бла бла бла бла бла.Бла бла бла бла бла бла бла бла бла бла. Бла бла бла бла бла бла бла бла бла бла. Бла бла бла бла бла бла бла бла бла бла. Бла бла бла бла бла бла бла бла бла бла. Бла бла бла бла бла бла бла бла бла бла. Бла бла бла бла бла бла бла бла бла бла.Бла бла бла бла бла бла бла бла бла бла. Бла бла бла бла бла бла бла бла бла бла. Бла бла бла бла бла бла бла бла бла бла. Бла бла бла бла бла бла бла бла бла бла. Бла бла бла бла бла бла бла бла бла бла. Бла бла бла бла бла бла бла бла бла бла.Бла бла бла бла бла бла бла бла бла бла. Бла бла бла бла бла бла бла бла бла бла. Бла бла бла бла бла бла бла бла бла бла. Бла бла бла бла бла бла бла бла бла бла.

\newpage

\section{Вторая часть}
\subsection{Первый кусок текста во второй части}

% \thispagestyle{empty}
% \pagestyle{empty}

Бла бла бла бла бла бла бла бла бла бла. Бла бла бла бла бла бла бла бла бла бла. Бла бла бла бла бла бла бла бла бла бла. Бла бла бла бла бла бла бла бла бла бла. Бла бла бла бла бла бла бла бла бла бла. Бла бла бла бла бла бла бла бла бла бла.Бла бла бла бла бла бла бла бла бла бла. Бла бла бла бла бла бла бла бла бла бла. Бла бла бла бла бла бла бла бла бла бла. Бла бла бла бла бла бла бла бла бла бла. Бла бла бла бла бла бла бла бла бла бла. Бла бла бла бла бла бла бла бла бла бла.Бла бла бла бла бла бла бла бла бла бла. Бла бла бла бла бла бла бла бла бла бла. Бла бла бла бла бла бла бла бла бла бла. Бла бла бла бла бла бла бла бла бла бла. Бла бла бла бла бла бла бла бла бла бла. Бла бла бла бла бла бла бла бла бла бла.Бла бла бла бла бла бла бла бла бла бла. Бла бла бла бла бла бла бла бла бла бла. Бла бла бла бла бла бла бла бла бла бла. Бла бла бла бла бла бла бла бла бла бла. Бла бла бла бла бла бла бла бла бла бла. Бла бла бла бла бла бла бла бла бла бла.Бла бла бла бла бла бла бла бла бла бла. Бла бла бла бла бла бла бла бла бла бла. Бла бла бла бла бла бла бла бла бла бла. Бла бла бла бла бла бла бла бла бла бла. Бла бла бла бла бла бла бла бла бла бла. Бла бла бла бла бла бла бла бла бла бла.

\subsection{Второй кусок текста во второй части}
Бла бла бла бла бла бла бла бла бла бла. Бла бла бла бла бла бла бла бла бла бла. Бла бла бла бла бла бла бла бла бла бла. Бла бла бла бла бла бла бла бла бла бла. Бла бла бла бла бла бла бла бла бла бла. Бла бла бла бла бла бла бла бла бла бла.Бла бла бла бла бла бла бла бла бла бла. Бла бла бла бла бла бла бла бла бла бла. Бла бла бла бла бла бла бла бла бла бла. Бла бла бла бла бла бла бла бла бла бла. Бла бла бла бла бла бла бла бла бла бла. Бла бла бла бла бла бла бла бла бла бла.Бла бла бла бла бла бла бла бла бла бла. Бла бла бла бла бла бла бла бла бла бла. Бла бла бла бла бла бла бла бла бла бла. Бла бла бла бла бла бла бла бла бла бла. Бла бла бла бла бла бла бла бла бла бла. Бла бла бла бла бла бла бла бла бла бла.Бла бла бла бла бла бла бла бла бла бла. Бла бла бла бла бла бла бла бла бла бла. Бла бла бла бла бла бла бла бла бла бла. Бла бла бла бла бла бла бла бла бла бла.


\newpage

\section{Третья часть}

%\thispagestyle{fancy}

\subsection{Первый кусок текста в третьей части}
Бла бла бла бла бла бла бла бла бла бла. Бла бла бла бла бла бла бла бла бла бла. Бла бла бла бла бла бла бла бла бла бла. Бла бла бла бла бла бла бла бла бла бла. Бла бла бла бла бла бла бла бла бла бла. Бла бла бла бла бла бла бла бла бла бла.Бла бла бла бла бла бла бла бла бла бла. Бла бла бла бла бла бла бла бла бла бла. Бла бла бла бла бла бла бла бла бла бла. Бла бла бла бла бла бла бла бла бла бла. Бла бла бла бла бла бла бла бла бла бла. Бла бла бла бла бла бла бла бла бла бла.Бла бла бла бла бла бла бла бла бла бла. Бла бла бла бла бла бла бла бла бла бла. Бла бла бла бла бла бла бла бла бла бла. Бла бла бла бла бла бла бла бла бла бла. 

\subsection{Второй кусок текста в третьей части}

Бла бла бла бла бла бла бла бла бла бла. Бла бла бла бла бла бла бла бла бла бла. Бла бла бла бла бла бла бла бла бла бла. Бла бла бла бла бла бла бла бла бла бла. Бла бла бла бла бла бла бла бла бла бла. Бла бла бла бла бла бла бла бла бла бла.Бла бла бла бла бла бла бла бла бла бла. 
\newpage




\end{document}