\documentclass[14pt,a4paper, oneside]{extreport}

%%%%%%%%%% Математика %%%%%%%%%%
\usepackage{amsmath,amsfonts,amssymb,amsthm,mathtools} 
%\mathtoolsset{showonlyrefs=true}  % Показывать номера только у тех формул, на которые есть \eqref{} в тексте.
%\usepackage{leqno} % Нумерация формул слева


%%%%%%%%%%%%%%%%%%%%%%%% Шрифты %%%%%%%%%%%%%%%%%%%%%%%%%%%%%%%%%
\usepackage{fontspec}         % пакет для подгрузки шрифтов
\setmainfont{Arial}   % задаёт основной шрифт документа

% why do we need \newfontfamily:
% http://tex.stackexchange.com/questions/91507/
\newfontfamily{\cyrillicfonttt}{Arial}
\newfontfamily{\cyrillicfont}{Arial}
\newfontfamily{\cyrillicfontsf}{Arial}


\usepackage{unicode-math}     % пакет для установки математического шрифта
\setmathfont{Asana Math}      % шрифт для математики
% \setmathfont[math-style=ISO]{Asana Math}
% Можно делать смену начертания с помощью разных стилей

% Конкретный символ из конкретного шрифта
% \setmathfont[range=\int]{Neo Euler}

\usepackage{polyglossia}      % Пакет, который позволяет подгружать русские буквы
\setdefaultlanguage{russian}  % Основной язык документа
\setotherlanguage{english}    % Второстепенный язык документа


%%%%%%%%%% Работа с картинками %%%%%%%%%
\usepackage{graphicx}                  % Для вставки рисунков
\usepackage{graphics} 
\graphicspath{{images/}{pictures/}}    % можно указать папки с картинками
\usepackage{wrapfig}                   % Обтекание рисунков и таблиц текстом


%%%%%%%%%% Работа с таблицами %%%%%%%%%%
\usepackage{tabularx}            % новые типы колонок
\usepackage{tabulary}            % и ещё новые типы колонок
\usepackage{array}               % Дополнительная работа с таблицами
\usepackage{longtable}           % Длинные таблицы
\usepackage{multirow}            % Слияние строк в таблице
\usepackage{float}               % возможность позиционировать объекты в нужном месте 
\usepackage{booktabs}            % таблицы как в книгах!  
\renewcommand{\arraystretch}{1.3} % больше расстояние между строками


%%%%%%%%%% Графика и рисование %%%%%%%%%%
\usepackage{tikz, pgfplots}  % язык для рисования графики из latex'a

%%%%%%%%%% Гиперссылки %%%%%%%%%%
\usepackage{xcolor}              % разные цвета

% Два способа включить в пакете какие-то опции:
%\usepackage[опции]{пакет}
%\usepackage[unicode,colorlinks=true,hyperindex,breaklinks]{hyperref}

\usepackage{hyperref}
\hypersetup{				
    unicode=true,           % позволяет использовать юникодные символы
    colorlinks=true,       	% true - цветные ссылки, false - ссылки в рамках
    urlcolor=blue,          % цвет ссылки на url
    linkcolor=red,          % внутренние ссылки
    citecolor=green,        % на библиографию
	pdfnewwindow=true,      % при щелчке в pdf на ссылку откроется новый pdf
	hyperindex,             % сделать ли ссылку кликабельной?
	breaklinks              % если ссылка не умещается в одну строку, разбивать ли ее на две части?   
}

\usepackage{csquotes}            % Еще инструменты для ссылок

%%%%%%%%%% Програмный код %%%%%%%%%%
\usepackage{minted}
% Включает подсветку команд в программах!
% Нужно, чтобы на компе стоял питон, надо поставить пакет Pygments, в котором он сделан, через pip.

% Для Windows: Жмём win+r, вводим cmd, жмём enter. Открывается консоль. 
% Прописываем easy_install Pygments
% Заходим в настройки texmaker и там прописываем в PdfLatex:
% pdflatex -shell-escape -synctex=1 -interaction=nonstopmode %.tex

% Для Linux: Открываем консоль. Убеждаемся, что у вас установлен pip командой pip --version
% Если он не установлен, ставим его: sudo apt-get install python-pip
% Ставим пакет sudo pip install Pygments

% После всего этого вы должны почувствовать себя тру-программистами! 
% Документация по пакету хорошая. Сам читал, погуглите!


%%%%%%%%%% Другие приятные пакеты %%%%%%%%%
\usepackage{multicol}       % несколько колонок
\usepackage{verbatim}       % для многострочных комментариев
\usepackage{makeidx}        % для создания предметных указателей

\usepackage{enumitem} % дополнительные плюшки для списков
%  например \begin{enumerate}[resume] позволяет продолжить нумерацию в новом списке

\usepackage{todonotes} % для вставки в документ заметок о том, что осталось сделать
% \todo{Здесь надо коэффициенты исправить}
% \missingfigure{Здесь будет Последний день Помпеи}
% \listoftodos --- печатает все поставленные \todo'шки



%%%% Оформление %%%%%%%
\usepackage{extsizes} % Возможность сделать 14-й шрифт
% размер листа бумаги
\usepackage[paper=a4paper,top=15mm, bottom=15mm,left=35mm,right=10mm,includefoot]{geometry}

\usepackage{indentfirst}       % установка отступа в первом абзаце главы!!!

\usepackage{setspace}  
\setstretch{1.33}  % Межстрочный интервал
\setlength{\parindent}{1.5em} % Красная строка.
\setlength{\parskip}{4mm}   % Расстояние между абзацами
% Разные длины в латехе https://en.wikibooks.org/wiki/LaTeX/Lengths

\flushbottom                            % Эта команда заставляет LaTeX чуть растягивать строки, чтобы получить идеально прямоугольную страницу
\righthyphenmin=2                       % Разрешение переноса двух и более символов
\widowpenalty=300                     % Небольшое наказание за вдовствующую строку (одна строка абзаца на этой странице, остальное --- на следующей)
\clubpenalty=3000                     % Приличное наказание за сиротствующую строку (омерзительно висящая одинокая строка в начале страницы)
\tolerance=1000     % Ещё какое-то наказание.


\usepackage{fancyhdr} % Колонтитулы
\pagestyle{fancy}

\renewcommand{\headrulewidth}{0.4pt}  % Толщина линий, отчеркивающих верхний
\renewcommand{\footrulewidth}{0.4pt}  % и нижний колонтитулы
	\lfoot{Нижний левый}
	\rfoot{Нижний правый}
	\rhead{Верхний правый}
 	\chead{Верхний в центре}
	\lhead{Верхний левый}
	\cfoot{ }         % Тут по умолчанию печатается номер страницы 
	%\lhead{\thepage} % Новая нумерация
	


\begin{document}



%%% Титульный лист!
    \pagestyle{fancy}
    \renewcommand{\headrulewidth}{0pt}
    \thispagestyle{empty}




\begin{titlepage}
\begin{center}
\small \bfseries Федеральное бюджетное образовательное учреждение \\
высшего образования\\
«РОССИЙСКАЯ АКАДЕМИЯ НАРОДНОГО ХОЗЯЙСТВА и\\
ГОСУДАРСТВЕННОЙ СЛУЖБЫ\\
при Президенте Российской Федерации»

\vspace{2ex}

ЭКОНОМИЧЕСКИЙ ФАКУЛЬТЕТ\\
НАПРАВЛЕНИЕ 38.03.01 ЭКОНОМИКА
\end{center}



\vfill


\noindent\small Группа ЭО-12-02


\hfill\parbox{0.45\linewidth}{
\parbox[t]{20em}{\centering\small
Кафедра Макроэкономики

\mbox{ }

\textbf{Допустить к защите}\\
заведующий кафедрой макроэкономики\\
\rule{8em}{0.5pt} Н.Л. Шагас\\
<<\rule{2em}{0.5pt}>> \rule{5em}{0.5pt} 201\rule{1em}{0.5pt} г. }}

\mbox{ }

\mbox{ }


\begin{center}\bfseries
ВЫПУСКНАЯ КВАЛИФИКАЦИОННАЯ РАБОТА

\mbox{ }

\large АНАЛИЗ ЭФФЕКТИВНОСТИ СИГНАЛОВ БАНКА РОССИИ\\
КАК ИНСТРУМЕНТА МОНЕТАРНОЙ ПОЛИТИКИ

\end{center}

\vfill


\noindent\normalsize
студент-бакалавр

\noindent
Ульянкин Филипп Валерьевич
\hfill /\rule{6em}{0.5pt}/\rule{6em}{0.5pt}/

\hfill\makebox[13em]{\hfill\footnotesize (подпись) \hfill\hfill (дата) \hfill}

\noindent
научный руководитель выпускной


\noindent
квалификационной работы к.э.н.,


\noindent
доцент кафедры микроэкономики

\noindent
Синельникова Елена Владимировна
\hfill /\rule{6em}{0.5pt}/\rule{6em}{0.5pt}/

\hfill\makebox[13em]{\hfill\footnotesize (подпись) \hfill\hfill (дата) \hfill}



%\noindent
%консультант
%
%\noindent
%д.э.н., профессор Петров Петр Петрович
%\hfill /\rule{6em}{0.5pt}/\rule{6em}{0.5pt}/
%
%\hfill\makebox[13em]{\hfill\footnotesize (подпись) \hfill\hfill (дата) \hfill}



\vfill
\vfill\vfill
\centering
\normalsize{\textbf{МОСКВА \\ 2016}}
\end{titlepage}




\newpage

\thispagestyle{empty}

\begin{center}
\small Российская академия народного хозяйства

 и государственной службы при Президенте РФ

 \rule{\linewidth}{0.5pt}

\vfill\normalsize НАУЧНО-ИССЛЕДОВАТЕЛЬСКАЯ РАБОТА НА ТЕМУ:

\vfill\normalsize


\Large\textsc{ \textbf{Написание сложных формул с закрытыми глазами и фальсификация расчётов так, чтобы всё было ок и тебя не спалили
}}\\

\vfill
\normalsize Москва\\ 2016

\end{center}

\newpage

%автоматическое оглавление
\tableofcontents

% Задать вопрос про то почему оно красное и как от этого избавиться.

\newpage

\section{Первая часть}
\subsection{Первый кусок текста в первой части}

Бла бла бла бла бла бла бла бла бла бла. Бла бла бла бла бла бла бла бла бла бла. Бла бла бла бла бла бла бла бла бла бла. Бла бла бла бла бла бла бла бла бла бла. Бла бла бла бла бла бла бла бла бла бла. Бла бла бла бла бла бла бла бла бла бла.Бла бла бла бла бла бла бла бла бла бла. Бла бла бла бла бла бла бла бла бла бла. Бла бла бла бла бла бла бла бла бла бла. Бла бла бла бла бла бла бла бла бла бла. Бла бла бла бла бла бла бла бла бла бла. Бла бла бла бла бла бла бла бла бла бла.Бла бла бла бла бла бла бла бла бла бла. Бла бла бла бла бла бла бла бла бла бла.

\subsection{Второй кусок текста в первой части}
Бла бла бла бла бла бла бла бла бла бла. Бла бла бла бла бла бла бла бла бла бла. Бла бла бла бла бла бла бла бла бла бла. Бла бла бла бла бла бла бла бла бла бла. Бла бла бла бла бла бла бла бла бла бла. Бла бла бла бла бла бла бла бла бла бла.Бла бла бла бла бла бла бла бла бла бла. Бла бла бла бла бла бла бла бла бла бла. Бла бла бла бла бла бла бла бла бла бла. Бла бла бла бла бла бла бла бла бла бла. Бла бла бла бла бла бла бла бла бла бла. Бла бла бла бла бла бла бла бла бла бла.Бла бла бла бла бла бла бла бла бла бла. Бла бла бла бла бла бла бла бла бла бла. Бла бла бла бла бла бла бла бла бла бла. Бла бла бла бла бла бла бла бла бла бла. Бла бла бла бла бла бла бла бла бла бла. Бла бла бла бла бла бла бла бла бла бла.Бла бла бла бла бла бла бла бла бла бла. Бла бла бла бла бла бла бла бла бла бла. Бла бла бла бла бла бла бла бла бла бла. Бла бла бла бла бла бла бла бла бла бла.

\newpage

\section{Вторая часть}
\subsection{Первый кусок текста во второй части}

% \thispagestyle{empty}
% \pagestyle{empty}

Бла бла бла бла бла бла бла бла бла бла. Бла бла бла бла бла бла бла бла бла бла. Бла бла бла бла бла бла бла бла бла бла. Бла бла бла бла бла бла бла бла бла бла. Бла бла бла бла бла бла бла бла бла бла. Бла бла бла бла бла бла бла бла бла бла.Бла бла бла бла бла бла бла бла бла бла. Бла бла бла бла бла бла бла бла бла бла. Бла бла бла бла бла бла бла бла бла бла. Бла бла бла бла бла бла бла бла бла бла. Бла бла бла бла бла бла бла бла бла бла. Бла бла бла бла бла бла бла бла бла бла.Бла бла бла бла бла бла бла бла бла бла. Бла бла бла бла бла бла бла бла бла бла. Бла бла бла бла бла бла бла бла бла бла. Бла бла бла бла бла бла бла бла бла бла. Бла бла бла бла бла бла бла бла бла бла. Бла бла бла бла бла бла бла бла бла бла.Бла бла бла бла бла бла бла бла бла бла. Бла бла бла бла бла бла бла бла бла бла. Бла бла бла бла бла бла бла бла бла бла. Бла бла бла бла бла бла бла бла бла бла. Бла бла бла бла бла бла бла бла бла бла. Бла бла бла бла бла бла бла бла бла бла.Бла бла бла бла бла бла бла бла бла бла. Бла бла бла бла бла бла бла бла бла бла. Бла бла бла бла бла бла бла бла бла бла. Бла бла бла бла бла бла бла бла бла бла. Бла бла бла бла бла бла бла бла бла бла. Бла бла бла бла бла бла бла бла бла бла.

\subsection{Второй кусок текста во второй части}
Бла бла бла бла бла бла бла бла бла бла. Бла бла бла бла бла бла бла бла бла бла. Бла бла бла бла бла бла бла бла бла бла. Бла бла бла бла бла бла бла бла бла бла. Бла бла бла бла бла бла бла бла бла бла. Бла бла бла бла бла бла бла бла бла бла.Бла бла бла бла бла бла бла бла бла бла. Бла бла бла бла бла бла бла бла бла бла. Бла бла бла бла бла бла бла бла бла бла. Бла бла бла бла бла бла бла бла бла бла. Бла бла бла бла бла бла бла бла бла бла. Бла бла бла бла бла бла бла бла бла бла.Бла бла бла бла бла бла бла бла бла бла. Бла бла бла бла бла бла бла бла бла бла. Бла бла бла бла бла бла бла бла бла бла. Бла бла бла бла бла бла бла бла бла бла. Бла бла бла бла бла бла бла бла бла бла. Бла бла бла бла бла бла бла бла бла бла.Бла бла бла бла бла бла бла бла бла бла. Бла бла бла бла бла бла бла бла бла бла. Бла бла бла бла бла бла бла бла бла бла. Бла бла бла бла бла бла бла бла бла бла.


\newpage

\section{Третья часть}

%\thispagestyle{fancy}

\subsection{Первый кусок текста в третьей части}
Бла бла бла бла бла бла бла бла бла бла. Бла бла бла бла бла бла бла бла бла бла. Бла бла бла бла бла бла бла бла бла бла. Бла бла бла бла бла бла бла бла бла бла. Бла бла бла бла бла бла бла бла бла бла. Бла бла бла бла бла бла бла бла бла бла.Бла бла бла бла бла бла бла бла бла бла. Бла бла бла бла бла бла бла бла бла бла. Бла бла бла бла бла бла бла бла бла бла. Бла бла бла бла бла бла бла бла бла бла. Бла бла бла бла бла бла бла бла бла бла. Бла бла бла бла бла бла бла бла бла бла.Бла бла бла бла бла бла бла бла бла бла. Бла бла бла бла бла бла бла бла бла бла. Бла бла бла бла бла бла бла бла бла бла. Бла бла бла бла бла бла бла бла бла бла. 

\subsection{Второй кусок текста в третьей части}

Бла бла бла бла бла бла бла бла бла бла. Бла бла бла бла бла бла бла бла бла бла. Бла бла бла бла бла бла бла бла бла бла. Бла бла бла бла бла бла бла бла бла бла. Бла бла бла бла бла бла бла бла бла бла. Бла бла бла бла бла бла бла бла бла бла.Бла бла бла бла бла бла бла бла бла бла. 
\newpage




\end{document}