\documentclass[12pt, a4paper]{article}  

\usepackage{etex} % расширение классического tex в частности позволяет подгружать гораздо больше пакетов, чем мы и займёмся далее

%%%%%%%%%% Математика %%%%%%%%%%
\usepackage{amsmath,amsfonts,amssymb,amsthm,mathtools} 
%\mathtoolsset{showonlyrefs=true}  % Показывать номера только у тех формул, на которые есть \eqref{} в тексте.
%\usepackage{leqno} % Нумерация формул слева


%%%%%%%%%%%%%%%%%%%%%%%% Шрифты %%%%%%%%%%%%%%%%%%%%%%%%%%%%%%%%%
\usepackage{fontspec}         % пакет для подгрузки шрифтов
\setmainfont{HelveticaNeueCyr}   % задаёт основной шрифт документа

% why do we need \newfontfamily:
% http://tex.stackexchange.com/questions/91507/
\newfontfamily{\cyrillicfonttt}{HelveticaNeueCyr}
\newfontfamily{\cyrillicfont}{HelveticaNeueCyr}
\newfontfamily{\cyrillicfontsf}{HelveticaNeueCyr}
% Иногда тех не видит структуры шрифтов. Эти трое бравых парней спасают ситуацию и доопределяют те куски, которые Тех не увидел.

\usepackage{unicode-math}     % пакет для установки математического шрифта
\setmathfont{Asana Math}      % шрифт для математики

\usepackage{polyglossia}      % Пакет, который позволяет подгружать русские буквы
\setdefaultlanguage{russian}  % Основной язык документа
\setotherlanguage{english}    % Второстепенный язык документа



%%%%%%%%%% Работа с картинками %%%%%%%%%
\usepackage{graphicx}                  % Для вставки рисунков
\usepackage{graphics} 
\graphicspath{{images/}{pictures/}}    % можно указать папки с картинками
\usepackage{wrapfig}                   % Обтекание рисунков и таблиц текстом
\usepackage{subfigure}                 % для создания нескольких рисунков внутри одного


%%%%%%%%%% Работа с таблицами %%%%%%%%%%
\usepackage{tabularx}            % новые типы колонок
\usepackage{tabulary}            % и ещё новые типы колонок
\usepackage{array}               % Дополнительная работа с таблицами
\usepackage{longtable}           % Длинные таблицы
\usepackage{multirow}            % Слияние строк в таблице
\usepackage{float}               % возможность позиционировать объекты в нужном месте 
\usepackage{booktabs}            % таблицы как в книгах!  
\renewcommand{\arraystretch}{1.3} % больше расстояние между строками


%%%%%%%%%% Гиперссылки %%%%%%%%%%
\usepackage{xcolor}              % разные цвета

% Два способа включить в пакете какие-то опции:
%\usepackage[опции]{пакет}
%\usepackage[unicode,colorlinks=true,hyperindex,breaklinks]{hyperref}

\usepackage{hyperref}
\hypersetup{				
    unicode=true,           % позволяет использовать юникодные символы
    colorlinks=true,       	% true - цветные ссылки, false - ссылки в рамках
    urlcolor=blue,          % цвет ссылки на url
    linkcolor=red,          % внутренние ссылки
    citecolor=green,        % на библиографию
	pdfnewwindow=true       % при щелчке в pdf на ссылку откроется новый pdf
	hyperindex=true         % сделать ли ссылку кликабельной?
	breaklinks=true         % если ссылка не умещается в одну строку, разбивать    
	                        % ли ее на две части?
}

\usepackage{csquotes}            % Еще инструменты для ссылок

%%%%%%%%%% Програмный код %%%%%%%%%%
\usepackage{minted}
% Включает подцветку комманд в программах!
% Нужно, чтобы на компе стоял питон, надо поставить пакет Pygments, в котором он сделан через pip или cmd ( нужно ввести easy_install Pygments ). 
% После нужно зайти в настройки texmaker и там прописать в PdfLatex pdflatex -shell-escape -synctex=1 -interaction=nonstopmode %.tex
% Документация по пакету хорошая, сам читал, погуглите!


%%%%%%%%%% Другие приятные пакеты %%%%%%%%%
\usepackage{multicol}       % несколько колонок
\usepackage{verbatim}       % для многострочных комментариев
\usepackage{mdframed}       % Этот пакет позволяет рисовать красивые рамки!
\usepackage{makeidx}        % для создания предметных указателей

\usepackage{enumitem} % дополнительные плюшки для списков
%  например \begin{enumerate}[resume] позволяет продолжить нумерацию в новом списке

\usepackage{todonotes} % для вставки в документ заметок о том, что осталось сделать
% \todo{Здесь надо коэффициенты исправить}
% \missingfigure{Здесь будет Последний день Помпеи}
% \listoftodos --- печатает все поставленные \todo'шки



\title{Оформление документа в целом}
\date{\today}



\begin{document} % конец преамбулы, начало документа

\maketitle

\section{Набор текста}

\subsection{Длинное тире и неразрывный пробел}

% --- это длинное тире 
Дима --- слесарь!

Дима - слесарь!

% ~ это неразрывный пробел
% Обычно неразрывный пробел ставится после предлогов, перед единицами измерения. В случае если 10 кг попадет на конец строки, ~ позволит сохранить их рядом, а не написать 10 на одной строке, а кг на другой.

\newpage


\subsection{Шрифт}

\begin{table}[h!]
	\caption{Размеры шрифта}
	\centering
		\begin{tabular}{|c|c|}
		\hline	\verb|\tiny|      & \tiny        крошечный \\
		\hline	\verb|\scriptsize|   & \scriptsize  очень маленький\\
			\hline \verb|\footnotesize| & \footnotesize  довольно маленький \\
			\hline \verb|\small|        &  \small        маленький \\
			\hline \verb|\normalsize|   &  \normalsize  нормальный \\
			\hline \verb|\large|        &  \large       большой \\
			\hline \verb|\Large|        &  \Large       еще больше \\[5pt]
			\hline \verb|\LARGE|        &  \LARGE       очень большой \\[5pt]
			\hline \verb|\huge|         &  \huge        огромный \\[5pt]
			\hline \verb|\Huge|         &  \Huge        громадный \\ \hline
		\end{tabular}
\end{table}

\begin{Huge}
\emph{Какой-нибудь \emph{обычный}  текст.}
\end{Huge}

Можно писать текст и \LARGE постоянно переключать \tiny шрифты между \normalsize собой.


\section{Сноски}

Чтобы сделать сноску к какому-то месту в тексте, достаточно использовать команду \verb|\footnote| с одним обязательным аргументом — текстом сноски. Cноски\footnote{Вроде этой.} нумеруются подряд на протяжении всей главы. 


\section{несколько колонок} 

Будут ли борелевскими на числовой прямой множества
\begin{multicols}{2}
\begin{enumerate}
    \item $(2;5)$,
    \item $(-\infty;t)$,
    \item $(t; +\infty)$,
    \item $[2;5]$,     
    \item $(-\infty;t]$,
    \item $[t; +\infty)$,
    \item $(3;5]$? 
\end{enumerate}
\end{multicols}

\begin{multicols}{3}
Какой-то длинный длинный текст, который в конечном счёте будет расположен  в нашем клёвом документе в три колонки. Три колонки --- это круто! Много колонок!!! Обожаю колонки!!! 
\end{multicols}


\section{Буквальное воспроизведение}

Окружение verbatim предназначено для буквального воспроизведения имеющихся в файле символов.

\begin{verbatim}
hсодержимое файла something.txti
	я                  мог бы 
писать      красиво, но $ }{%        пишу 

криво!
\end{verbatim}

Можно использовать это окружения для написания команд напрямую. Например, команда \verb"\dots" задает многоточие.

\newpage
\section{Красивое оформление кода}

Окружение minted красиво и с автоматической подцветкой оформляет код в питоне!

\begin{minted}[breaklines]{python}
import numpy as np
import pandas as pd

#Вспомогательная функция для получения правильного количества дней. Работает даже с високосным годом.
def monthlength(month,year):
    if year % 4 == 0:
         VisYear = 29
    else:
         VisYear = 28
    return [31,VisYear,31,30,31,30,31,31,30,31,30,31][month]
\end{minted} 

Окружение minted красиво и с автоматической подцветкой оформляет код в R! При этом оно делает это с разной уникальной подцветкой, соответствующей оригинальной подцветке языка! 

\begin{minted}[breaklines,linenos]{R}
library("dplyr")
library("ggplot2")

ALLdata <- read.csv("data.txt",sep="\t",dec=".",header=TRUE)
data_usd <- data.frame(ALLdata$STATEMENT[-1],diff(ALLdata$USD,1))
names(data_usd) <- c('Statement','USD')
\end{minted} 

Можно писать математические комменатрии к коду.

\begin{minted}[mathescape]{python}
# Код ниже выдаст $\sum_{i=1}^{n}i$
def sum_from_one_to(n):
	r = range(1, n + 1)
	return sum(r)
\end{minted}

\newpage
\section{Гиперссылки}

\url{https://vk.com}

В \href{https://vk.com}{этой социальной сети} можно многое найти!

\section{Todo и Missfigure}

\todo{Здесь надо коэффициенты исправить}
\missingfigure{Здесь будет Последний день Помпеи} 


\end{document} % конец документа