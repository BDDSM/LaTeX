%!TEX TS-program = xelatex
\documentclass[12pt, a4paper, oneside]{article}

\usepackage{amsmath,amsfonts,amssymb,amsthm,mathtools}  % пакеты для математики

\usepackage[utf8]{inputenc} % задание utf8 кодировки исходного tex файла
\usepackage[british,russian]{babel} % выбор языка для документа
\usepackage[X2,T2A]{fontenc}        % кодировка

\usepackage{fontspec}         % пакет для подгрузки шрифтов
\setmainfont{Helvetica}   % задаёт основной шрифт документа

% why do we need \newfontfamily:
% http://tex.stackexchange.com/questions/91507/
\newfontfamily{\cyrillicfonttt}{Helvetica}
\newfontfamily{\cyrillicfont}{Helvetica}
\newfontfamily{\cyrillicfontsf}{Helvetica}

\usepackage{unicode-math}     % пакет для установки математического шрифта
\setmathfont{Neo Euler}      % шрифт для математики
% \setmathfont[math-style=ISO]{Asana Math}
% Можно делать смену начертания с помощью разных стилей

% Конкретный символ из конкретного шрифта
% \setmathfont[range=\int]{Neo Euler}

%%%%%%%%%% Работа с картинками %%%%%%%%%
\usepackage{graphicx}                  % Для вставки рисунков
\usepackage{graphics}
\graphicspath{{images/}{pictures/}}    % можно указать папки с картинками
\usepackage{wrapfig}                   % Обтекание рисунков и таблиц текстом

%%%%%%%%%%%%%%%%%%%%%%%% Графики и рисование %%%%%%%%%%%%%%%%%%%%%%%%%%%%%%%%%
\usepackage{tikz, pgfplots}  % язык для рисования графики из latex'a

%%%%%%%%%% Гиперссылки %%%%%%%%%%
\usepackage{xcolor}              % разные цвета

\usepackage{hyperref}
\hypersetup{
	unicode=true,           % позволяет использовать юникодные символы
	colorlinks=true,       	% true - цветные ссылки, false - ссылки в рамках
	urlcolor=blue,          % цвет ссылки на url
	linkcolor=red,          % внутренние ссылки
	citecolor=green,        % на библиографию
	pdfnewwindow=true,      % при щелчке в pdf на ссылку откроется новый pdf
	breaklinks              % если ссылка не умещается в одну строку, разбивать ли ее на две части?
}


\usepackage{todonotes} % для вставки в документ заметок о том, что осталось сделать
% \todo{Здесь надо коэффициенты исправить}
% \missingfigure{Здесь будет Последний день Помпеи}
% \listoftodos --- печатает все поставленные \todo'шки

\usepackage{enumitem} % дополнительные плюшки для списков
%  например \begin{enumerate}[resume] позволяет продолжить нумерацию в новом списке

\usepackage[paper=a4paper, top=20mm, bottom=15mm,left=20mm,right=15mm]{geometry}
\usepackage{indentfirst}       % установка отступа в первом абзаце главы

\usepackage{setspace}
\setstretch{1.15}  % Межстрочный интервал
\setlength{\parskip}{4mm}   % Расстояние между абзацами
% Разные длины в латехе https://en.wikibooks.org/wiki/LaTeX/Lengths


\usepackage{xcolor} % Enabling mixing colors and color's call by 'svgnames'

\definecolor{MyColor1}{rgb}{0.2,0.4,0.6} %mix personal color
\newcommand{\textb}{\color{Black} \usefont{OT1}{lmss}{m}{n}}
\newcommand{\blue}{\color{MyColor1} \usefont{OT1}{lmss}{m}{n}}
\newcommand{\blueb}{\color{MyColor1} \usefont{OT1}{lmss}{b}{n}}
\newcommand{\red}{\color{LightCoral} \usefont{OT1}{lmss}{m}{n}}
\newcommand{\green}{\color{Turquoise} \usefont{OT1}{lmss}{m}{n}}

\usepackage{titlesec}
\usepackage{sectsty}
%%%%%%%%%%%%%%%%%%%%%%%%
%set section/subsections HEADINGS font and color
\sectionfont{\color{MyColor1}}  % sets colour of sections
\subsectionfont{\color{MyColor1}}  % sets colour of sections

%set section enumerator to arabic number (see footnotes markings alternatives)
\renewcommand\thesection{\arabic{section}.} %define sections numbering
\renewcommand\thesubsection{\thesection\arabic{subsection}} %subsec.num.

%define new section style
\newcommand{\mysection}{
	\titleformat{\section} [runin] {\usefont{OT1}{lmss}{b}{n}\color{MyColor1}} 
	{\thesection} {3pt} {} } 


%	CAPTIONS
\usepackage{caption}
\usepackage{subcaption}
%%%%%%%%%%%%%%%%%%%%%%%%
\captionsetup[figure]{labelfont={color=Turquoise}}

\usepackage[normalem]{ulem}  % для зачекивания текста

\pagestyle{empty}

\begin{document}
	
\section*{Задание 2 :  Тот, чьё имя никто не знает (15 баллов)}

Поздравляю! Вы тот, кто рассылает письма из Хогвартса! И вы столкнулись с проблемой. Министерство \sout{Образования РФ}  Магии недавно приняло совершенно дурацкий законопроект, который задаёт единый образец для оформления всех идущих из Хогвартса писем. Как же хорошо, что в мире волшебников есть \LaTeX{}, недоступный для маглов. Ведь он поможет один раз и навсегда написать такой шаблон. 

\textbf{Требования к письму из школы:}
\begin{enumerate}
\item Текст письма должен быть написан на листе формата A4
\item Текст печатается с единичным интервалом между строками шрифтом Arial 14 размера
\item Поля страницы должны быть отрегулированны следующим образом: отступы сверху и снизу листа по 10мм, слева и справа по 35мм
\item Никаких красных строк в начале абзаца, расстояние между абзацами 4мм
\item  Наверху, перед текстом письма, ровно по центру должен быть расположен герб школы чародейства и волшебства. Высота картинки должна быть ровно 6 см.
\item  Ровно через 1 см от герба расположена информация о том кому именно предназначено это письмо. Например, "Мистеру Поттеру". Эта информация набрана 12 шрифтом.
\item  Ровно через 2.5 см от информации о получателе начинается текст письма. Текст может быть любым. Главное, чтобы в нём было отражено, что получатель приглашается для обучения в Хогвартс.
\item  Информация, которая идёт после текста письма должна быть всегда прибита к низу страницы вне зависимости от того насколько длинным получилось письмо.
\item Подпись профессора МакГонагалл может быть выполнена любым шрифтом, имитирующим человеческую подпись. Также она может быть прорисована в TiKz подобно xkcd стилю.
\item  В самом низу есть гиперссылка на сайт Школы Чародейства и Волшебства.
\item Страница с письмом не нумеруется.
\end{enumerate}


\textbf{Требования к приложению:}
\begin{enumerate}[resume]
\item  В приложении А располагаются список необходимых книг и предметов.  Приложения нумеруются русскими буквами. Перед номером приложения идёт слово "Приложение".
\item  Каждый новый предмет идёт после изображения волшебной палочки.
\item  В приложении Б располагается список предметов, изучаемых в течение первого учебного года.
\item  На каждой странице расположены колонтитулы, которые необходимо оформить также как в образце.
\end{enumerate}

При оформлении письма оставьте в преамбуле только те пакеты, которые реально используются в документе.

\begin{itemize}
\item[$(10)$] Вы оформили письмо в соответствии с требованиями.

\item[$(5)$] Вы отправили письмо другу, который не ходит на факультатив. В качестве подтверждения факта отправки прикреплён скрин с реакцией друга на письмо.
\end{itemize}

Итоговый pdf-файл, tex-файл и все картинки, которые использовались в документе, нужно положить в папку на свой Dropbox, Github, yandex-disk или другой репозиторий. После нужно заполнить \href{https://docs.google.com/forms/d/e/1FAIpQLSe11kxKVfv07iCL1E9yNX7ll9swKImiVwRr1H70lslGzInRSg/viewform}{уютную гугл-форму.}  

Если вы выполняете работу в overleaf, то прикрепляйте в форму ссылку на неё. Если вы прикрепите ссылку с возможностью редактирования вашего файла, при проверке мы оставим у вас в файле полезные комментарии. Если вы кидаете ссылку на репозиторий, все комменты отправим в личку.

Перед этим обязательно проверьте компилируется ли файл без ошибок на вашем компьютере. Не стесняйтесь абсолютно в любое время дня и ночи просить о помощи, если она вам действительно необходима! \textbf{Также не забывайте про то, что любое творчество поощряется,} а ещё не забываете, где  находится  \href{https://fulyankin.github.io/LaTeX/}{страничку курса} с кучей шпаргалок!

\end{document}
