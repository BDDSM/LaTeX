\documentclass[12pt, a4paper]{article}  % Любой документ начинается с такой строки! В ней мы выбираем размер шрифта, размер бумаги и класс документа. У каждого класса свои свойства!

% Знак процента используется для комментариев. Все, что написано под знаком процента, LaTeX не видит. 
%
%         Классы: 
% article   ---   статья
% report    ---   отчет
% book      ---   книга
% beamer    ---   презентация
%
% Каждый документ состоит из двух частей. Часть от \documentclass  до \begin{document} - преамбула. Часть до \end{document} - тело документа.
%
% В преамбуле находятся различные служебные команды. А именно:
% а) Команды, подключающие пакеты
% б) Команды, которые определяют вид документа в целом
% в) Команды, которые создают новые команды, чтобы удобнее использовать старые команды!
% г) Ещё какие-нибудь другие команды

\usepackage{amsmath,amsfonts,amssymb,amsthm,mathtools}  % Тут мы подключаем пакеты для математики!


%%%%%%%%%%%%%%%%%%%%%%%% Шрифты %%%%%%%%%%%%%%%%%%%%%%%%%%%%%%%%%

\usepackage{fontspec}         % пакет для подгрузки шрифтов
\setmainfont{Roboto}          % задаёт основной шрифт документа

\usepackage{unicode-math}     % пакет для установки математического шрифта
\setmathfont{Asana Math}      % шрифт для математики

\usepackage{polyglossia}      % Пакет, который позволяет подгружать русские буквы
\setdefaultlanguage{russian}  % Основной язык документа
\setotherlanguage{english}    % Второстепенный язык документа

% Внимание! Все, кто использует кодировку cp1251 будут гореть в аду! Используйте только utf-8! 


\begin{document} % Тут заканчиваются служебные команды и начинается документ!

\section*{Четыре благородные истины}

\begin{enumerate}
\item Жизнь полна страданий.
\item Причина страданий сама жизнь с её страстями и желаниями.
\item Человек может устранить причину своих страданий, обуздав свои страсти и желания.
\item Есть путь, который ведёт к освобождению. И изучение \LaTeX{ } часть этого пути.
\end{enumerate}

\end{document}
