\documentclass[10pt,a4paper]{article}
\usepackage[landscape]{geometry}
\usepackage{etex} % расширение классического tex в частности позволяет подгружать гораздо больше пакетов, чем мы и займёмся далее

%%%%%%%%%% Математика %%%%%%%%%%
\usepackage{amsmath,amsfonts,amssymb,amsthm,mathtools}
%\mathtoolsset{showonlyrefs=true}  % Показывать номера только у тех формул, на которые есть \eqref{} в тексте.
%\usepackage{leqno} % Нумерация формул слева


%%%%%%%%%%%%%%%%%%%%%%%% Шрифты %%%%%%%%%%%%%%%%%%%%%%%%%%%%%%%%%
\usepackage{fontspec}         % пакет для подгрузки шрифтов
\setmainfont{Roboto}   % задаёт основной шрифт документа

% why do we need \newfontfamily:
% http://tex.stackexchange.com/questions/91507/
\newfontfamily{\cyrillicfonttt}{Roboto}
\newfontfamily{\cyrillicfont}{Roboto}
\newfontfamily{\cyrillicfontsf}{Roboto}
% Иногда тех не видит структуры шрифтов. Эти трое бравых парней спасают ситуацию и доопределяют те куски, которые Тех не увидел.

\usepackage{unicode-math}     % пакет для установки математического шрифта
\setmathfont{Asana Math}      % шрифт для математики

\usepackage{polyglossia}      % Пакет, который позволяет подгружать русские буквы
\setdefaultlanguage{russian}  % Основной язык документа
\setotherlanguage{english}    % Второстепенный язык документа



%%%%%%%%%% Работа с картинками %%%%%%%%%
\usepackage{graphicx}                  % Для вставки рисунков
\usepackage{graphics}
\graphicspath{{images/}{pictures/}}    % можно указать папки с картинками
\usepackage{wrapfig}                   % Обтекание рисунков и таблиц текстом
\usepackage{subfigure}                 % для создания нескольких рисунков внутри одного


%%%%%%%%%% Работа с таблицами %%%%%%%%%%
\usepackage{tabularx}            % новые типы колонок
\usepackage{tabulary}            % и ещё новые типы колонок
\usepackage{array}               % Дополнительная работа с таблицами
\usepackage{longtable}           % Длинные таблицы
\usepackage{multirow}            % Слияние строк в таблице
\usepackage{float}               % возможность позиционировать объекты в нужном месте
\usepackage{booktabs}            % таблицы как в книгах!
\renewcommand{\arraystretch}{1.3} % больше расстояние между строками

\begin{document}

\begin{table}
\begin{tabulary}{\textwidth}{|L|L|L|L|L|L|}
\hline
Метод & Оценка & Год создания & Автор & Фотка автора & Описание \\
\hline
Метод наименьших квадратов (OLS) & $(X^{T}X)^{-1}X^Ty$ & 1795, 1805 & Гаусс, Лежандр &  Gauss, Lezhandr & Метод оценивания параметров эконометрической модели, состоящий в минимизации суммы квадратов расхождений между наблюдаемыми значениями зависимой переменной и значениями этой переменной, вычисленными для наблюдаемых значений независимых переменных по оценённой модели связи. \\
\hline
Взвешенный метод наименьших квадратов (WLS) &  $(X^T X)^{-1}X^T y$ &  хз & хз & хз & Процедура, состоящая в минимизации определённым образом взвешенной суммы квадратов отклонений наблюдаемых значений зависиммой переменной от значений, вычисляемых по подбираемой модели связи. \\
\hline
Обобщённый метод наименьших квадратов (GLS) &  та же & хз & хз & хз & Теоретическая процедура оценивания коэффициентов линенйиной модели регрессии в ситуации, когда случайные ошибки имеют разные дисперсии и коррелированы между собой, при этом предполагается,  что ковариационная матрица вектора ошибок невырождена и все ее элементы известны. \\
\hline
Доступный обобщённый метод наименьших квадратов (FGLS) & ла ла & хз &  & & Практически реализуемая процедура оценивания коэффициентов линейной модели регрессии в ситуации, когда случайные ошибки имеют разные дисперсии и коррелированы между собой, повторяющая процедуру обобщенного метода наисеньших квадратов, но импользующая оцененную ковариационную матрицу вектора ошибок. \\
\hline
Косвенный метод наименьших квадратов (ILS) & & & & & метод получения оценок параметров $i-$го стохастического уравнения структурной формы через оценки наименьших квадратов коэффициентов уравнений приведенной формы. Метод применим в случае точной идентифицируемости $i-$го структурного уравнения.\\
\hline
Двухшаговый метод наименьших квадратов (2SLS) & & & & & Метод оценивания коэффициентов уравнения структурной формы, состоящий в предварительной очистке стохастической объясняющей переменой от коррелированности с ошибкой в этом уравнении с использованием инструментальных переменных и в последующем оценивании уравнения, в котором исходная объясняющая переменная заменяется ее очищенным вариантом. \\
\hline
Трёхшаговый метод наименьших квадратов (3SLS) & & & & & Доступный обобщённый метод наименьших квадратов, применённый к системе одновременных уравнений. Принимает во внимание наличие коррелирванности между ошибками в разных структурных уравнениях.
\hline
Динамический метод наименьших квадратов (DOLS) & & & & & Процендура, уменьшающая смещение OLS- оыенок коэффициентов коинтегрирующей регрессии. В случае когда система состоит из $I(1)-$рядов (имеющих порядок интегрированности 1), сводится к добавлению в правую часть уравнения текущих, запаздывающих и опережающих разностей всех объясняющих переменных (метод leads-and-lags).\\
\hline
Динамисеский обобщённый метод наименьших квадратов (DGLS) & & & & & Вариант DOLS, учитывающий автокоррелированность в остатках расширенного уравнения DOLS.
\hline


\end{tabulary}
\end{table}

\end{document}
