%!TEX TS-program = xelatex
\documentclass[12pt, a4paper]{article}  

\usepackage{etex} % расширение классического tex в частности позволяет подгружать гораздо больше пакетов, чем мы и займёмся далее

%%%%%%%%%% Математика %%%%%%%%%%
\usepackage{amsmath,amsfonts,amssymb,amsthm,mathtools} 
%\mathtoolsset{showonlyrefs=true}  % Показывать номера только у тех формул, на которые есть \eqref{} в тексте.
%\usepackage{leqno} % Нумерация формул слева


%%%%%%%%%%%%%%%%%%%%%%%% Шрифты %%%%%%%%%%%%%%%%%%%%%%%%%%%%%%%%%
\usepackage{fontspec}         % пакет для подгрузки шрифтов
\setmainfont{Arial}   % задаёт основной шрифт документа

\defaultfontfeatures{Mapping=tex-text}

% why do we need \newfontfamily:
% http://tex.stackexchange.com/questions/91507/
\newfontfamily{\cyrillicfonttt}{Arial}
\newfontfamily{\cyrillicfont}{Arial}
\newfontfamily{\cyrillicfontsf}{Arial}

\usepackage{unicode-math}     % пакет для установки математического шрифта
\setmathfont{Asana Math}      % шрифт для математики
% \setmathfont[math-style=ISO]{Asana Math}
% Можно делать смену начертания с помощью разных стилей

% Конкретный символ из конкретного шрифта
% \setmathfont[range=\int]{Neo Euler}

\usepackage{polyglossia}      % Пакет, который позволяет подгружать русские буквы
\setdefaultlanguage{russian}  % Основной язык документа
\setotherlanguage{english}    % Второстепенный язык документа


%%%%%%%%%% Работа с картинками %%%%%%%%%
\usepackage{graphicx}                  % Для вставки рисунков
\usepackage{graphics}
\graphicspath{{images/}{pictures/}}    % можно указать папки с картинками
\usepackage{wrapfig}                   % Обтекание рисунков и таблиц текстом
\usepackage{subfigure}                 % для создания нескольких рисунков внутри одного


%%%%%%%%%% Работа с таблицами %%%%%%%%%%
\usepackage{tabularx}            % новые типы колонок
\usepackage{tabulary}            % и ещё новые типы колонок
\usepackage{array}               % Дополнительная работа с таблицами
\usepackage{longtable}           % Длинные таблицы
\usepackage{multirow}            % Слияние строк в таблице
\usepackage{float}               % возможность позиционировать объекты в нужном месте
\usepackage{booktabs}            % таблицы как в книгах!
\renewcommand{\arraystretch}{1.3} % больше расстояние между строками

% Заповеди из документации к booktabs:
% 1. Будь проще! Глазам должно быть комфортно
% 2. Не используйте вертикальные линни
% 3. Не используйте двойные линии. Как правило, достаточно трёх горизонтальных линий
% 4. Единицы измерения - в шапку таблицы
% 5. Не сокращайте .1 вместо 0.1
% 6. Повторяющееся значение повторяйте, а не говорите "то же"
% 7. Есть сомнения? Выравнивай по левому краю!

%%%%%%%%%% Графика и рисование %%%%%%%%%%
\usepackage{tikz, pgfplots}  % язык для рисования графики из latex'a
\usepackage{amscd}                  %Пакеты для рисования
\usepackage[matrix,arrow,curve]{xy} %комунитативных диаграмм

% Всякие странные команды из Geogebra и с сайта для TikZ
\usepackage{pgf}
\usepackage{mathrsfs}
\usetikzlibrary{arrows}
\pagestyle{empty}


\usetikzlibrary{calc}
\usepackage{relsize}
\newcommand\LM{\ensuremath{\mathit{LM}}}
\newcommand\IS{\ensuremath{\mathit{IS}}}

\usepackage{pgf,tikz}
\usepackage{mathrsfs}
\usetikzlibrary{arrows}
\pagestyle{empty}

\title{Домашнее задание 3. \\ Упражнение 2}
\author{Решетникова Дарья}
\date{\today}

\begin{document}
	
\maketitle
	
\definecolor{xdxdff}{rgb}{0.49019607843137253,0.49019607843137253,1.}
\definecolor{zzttqq}{rgb}{0.6,0.2,0.}
\definecolor{qqqqff}{rgb}{0.,0.,1.}
\definecolor{cqcqcq}{rgb}{0.7529411764705882,0.7529411764705882,0.7529411764705882}

\newenvironment{kot}[1]{
\definecolor{zzttqq}{rgb}{0.6,0.2,0.}

\begin{center}
	
\begin{tikzpicture}[line cap=round,line join=round,>=triangle 45,x=1.0cm,y=1.0cm]

\node[draw,text width=3cm] at (13.05, 14.25) {#1};}
{
\draw [color=cqcqcq,, xstep=1.0cm,ystep=1.0cm] (1.1089078250735587,3.600216385884293) ;
\clip(1.1089078250735587,3.600216385884293) rectangle (16.67611712373211,16.061207587826747);
\fill[color=zzttqq,fill=zzttqq,fill opacity=0.10000000149011612] (7.,5.) -- (6.46,4.22) -- (7.36,4.22) -- cycle;
\fill[color=zzttqq,fill=zzttqq,fill opacity=0.10000000149011612] (9.,5.) -- (8.44,4.26) -- (9.36,4.26) -- cycle;
\fill[color=zzttqq,fill=zzttqq,fill opacity=0.10000000149011612] (7.4042148038092925,12.33438026158761) -- (7.,13.) -- (7.051850780299562,11.834979757240136) -- cycle;
\fill[color=zzttqq,fill=zzttqq,fill opacity=0.10000000149011612] (8.582844316200722,12.343923194759617) -- (9.,13.) -- (8.902108604310815,11.954969957982975) -- cycle;
\draw [rotate around={90.:(8.,7.5)}] (8.,7.5) ellipse (3.0495097567963945cm and 1.7462845577958925cm);
\draw [color=zzttqq] (7.,5.)-- (6.46,4.22);
\draw [color=zzttqq] (6.46,4.22)-- (7.36,4.22);
\draw [color=zzttqq] (7.36,4.22)-- (7.,5.);
\draw [color=zzttqq] (9.,5.)-- (8.44,4.26);
\draw [color=zzttqq] (8.44,4.26)-- (9.36,4.26);
\draw [color=zzttqq] (9.36,4.26)-- (9.,5.);
\draw(8.,11.54) circle (0.9929753269845124cm);
\draw [color=zzttqq] (7.4042148038092925,12.33438026158761)-- (7.,13.);
\draw [color=zzttqq] (7.,13.)-- (7.051850780299562,11.834979757240136);
\draw [color=zzttqq] (7.051850780299562,11.834979757240136)-- (7.4042148038092925,12.33438026158761);
\draw [color=zzttqq] (8.582844316200722,12.343923194759617)-- (9.,13.);
\draw [color=zzttqq] (9.,13.)-- (8.902108604310815,11.954969957982975);
\draw [color=zzttqq] (8.902108604310815,11.954969957982975)-- (8.582844316200722,12.343923194759617);
\draw (6.487521207097226,9.024295027671684)-- (4.64,10.66);
\draw (9.504475462752403,9.048259610351636)-- (11.32,10.7);
\draw (9.560710553347574,11.4741180729088)-- (10.228908632292315,12.0881379292364);
\draw (10.62621559815135,12.413207264939247)-- (11.330532492174184,13.045286528805892);
\draw (12.251562276665585,13.839900460523962)-- (11.818136495728455,13.424534087125881);

\begin{scriptsize}
\draw [fill=qqqqff] (7.538056908974306,11.835306223689742) circle (2.5pt);
\draw [fill=qqqqff] (8.4952055085438,11.853365631228789) circle (2.5pt);
\draw [fill=qqqqff] (8.,11.) circle (2.5pt);
\end{scriptsize}
\end{tikzpicture}
\end{center}
}

\begin{kot}
{Hello!}
\end{kot}
\end{document}