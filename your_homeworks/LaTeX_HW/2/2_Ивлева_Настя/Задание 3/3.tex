\documentclass[12pt, a4paper]{article}  


\usepackage{amsmath,amsfonts,amssymb,amsthm,mathtools} 

\usepackage{leqno}



\usepackage{fontspec}   
\setmainfont{Arial}
\defaultfontfeatures{Mapping=tex-text}
\newfontfamily{\cyrillicfonttt}{Arial}
\newfontfamily{\cyrillicfont}{Arial}
\newfontfamily{\cyrillicfontsf}{Arial}

\usepackage{unicode-math}     
\setmathfont{Asana Math}      

\usepackage{polyglossia}      
\setdefaultlanguage{russian}  
\setotherlanguage{english}    


\author{Ивлева Анастасия Владимировна} 
\title{Домашняя работа №2}


\begin{document}
\maketitle
\section{Письмо с угрозой}

\Large{\fontspec{Phorssa}{Дорогие и Наши Любимые преподаватели!Не задавайте нам много домашней работы и будьте к нам дружелюбными. Иначе хоть мы и младше вас, но сумеем отомстить и купим вам кучу вкусняшек. И тогда вы потолстеете, но хоть будете милыми... :)Конец угрозы.}}

\section{Формулы}
\setmathfont{Phorssa}
\Large{\fontspec{Phorssa}{\[ 9 + 9 = 18 \]}
\Large{\fontspec{Phorssa}{\[(X^T \Omega^{-1} X)^{-1}X^T \Omega^{-1} y\]}

\end{document}