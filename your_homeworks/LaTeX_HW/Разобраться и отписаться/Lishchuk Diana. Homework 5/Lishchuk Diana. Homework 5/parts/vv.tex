% !TEX root = ../Lishchuk Diana . Homework 5.1.tex

	\chapter*{Введение}
%Включение введения в соодержание
\addcontentsline{toc}{chapter}{Введение}

Во многих научных работах зарубежных и российских авторов King, R.G.and  R.Levine, Allen F., Hellwig M., М.Столбова и других показано, что уровень развития финансовой системы влияет на уровень экономического развития. Россия представляет собой страну с большой территорией, разделенную на несколько десятков субъектов, в каждом из которых действует множество коммерческих банков. Субъекты РФ различаются между собой по уровню экономического развития, одной из причин этого факта может являться различный уровень развития региональной банковской системы.

Региональная банковская система представляет собой совокупность субъектов финансового сектора, которые проводят банковские операции на территории соответствующего региона, организации, осуществляющей деятельность Банка России в данном регионе страны, и других региональных некоммерческих финансовых организаций. Для нормального функционирования и развития экономики региона необходимо наличие банковского сектора с высоким уровнем развития, способного осуществлять финансирование деятельности предприятий, оказывать поддержку малому и среднему предпринимательству и частному сектору. Другими словами, региональная банковской системе необходимо расширять возможности кредитования реального сектора экономики, что должно положительно влиять не только на банковский сектор региона, но и  банковскую систему России. Недостаточный уровень развития региональной финансовой системы может приводить к замедлению темпов производственной деятельности, снижение деловой активности, сдерживанию темпов инновационной деятельности вследствие ограниченности ресурсов, которые могут взять фирмы у банков, снижению стимулов для открытия нового бизнеса и т.д. 

Целью данной работы является выявление связи между банковским сектором и ростом реального сектора экономики регионов. Для достижения поставленной цели необходимо осуществить ряд задач:
\begin{enumerate}
	\item с теоретической точки зрения обосновать влияние банковского сектора на экономический рост;
	\item изучить эконометрические методы оценивания связи между показателями развития банковского сектора и реальным сектором;
	\item Определить с помощью каких показателей можно характеризовать уровень развития банковского сектора;
	\item Собрать необходимые данные для проведения собственного эконометрического исследования;
	\item	На основе отобранных показателей с помощью эконометрических методов оценить, каким образом уровень развития банковской системы влияет на экономическое положение регионов РФ;
	\item	Сформулировать выводы о влиянии банковского сектора на экономический рост в регионах.
\end{enumerate} 
\missingfigure{Немного выводов по итогу работы!}
