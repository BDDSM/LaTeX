\documentclass[12pt, a4paper]{article}

\usepackage{xcolor}

%%%%%%%%%% Програмный код %%%%%%%%%%
\usepackage{minted}
% Включает подсветку команд в программах!
% Нужно, чтобы на компе стоял питон, надо поставить пакет Pygments, в котором он сделан, через pip.

% Для Windows: Жмём win+r, вводим cmd, жмём enter. Открывается консоль.
% Прописываем easy_install Pygments
% Заходим в настройки texmaker и там прописываем в PdfLatex:
% pdflatex -shell-escape -synctex=1 -interaction=nonstopmode %.tex

% Для Linux: Открываем консоль. Убеждаемся, что у вас установлен pip командой pip --version
% Если он не установлен, ставим его: sudo apt-get install python-pip
% Ставим пакет sudo pip install Pygments

% Для Mac: Всё то же самое, что на Linux, но через brew.

% После всего этого вы должны почувствовать себя тру-программистами!
% Документация по пакету хорошая. Сам читал, погуглите!


%%%%%%%%%% Математика %%%%%%%%%%
\usepackage{amsmath,amsfonts,amssymb,amsthm,mathtools}
%\mathtoolsset{showonlyrefs=true}  % Показывать номера только у тех формул, на которые есть \eqref{} в тексте.
%\usepackage{leqno} % Нумерация формул слева


%%%%%%%%%%%%%%%%%%%%%%%% Шрифты %%%%%%%%%%%%%%%%%%%%%%%%%%%%%%%%%
\usepackage{fontspec}         % пакет для подгрузки шрифтов
\setmainfont{Arial}   % задаёт основной шрифт документа

% Команда, которая нужна для корректного отображения длинных тире и некоторых других символов.
\defaultfontfeatures{Mapping=tex-text}

% why do we need \newfontfamily:
% http://tex.stackexchange.com/questions/91507/
\newfontfamily{\cyrillicfonttt}{Arial}
\newfontfamily{\cyrillicfont}{Arial}
\newfontfamily{\cyrillicfontsf}{Arial}

\usepackage{unicode-math}     % пакет для установки математического шрифта
\setmathfont{Asana Math}      % шрифт для математики
% \setmathfont[math-style=ISO]{Asana Math}
% Можно делать смену начертания с помощью разных стилей

% Конкретный символ из конкретного шрифта
% \setmathfont[range=\int]{Neo Euler}

\usepackage{polyglossia}      % Пакет, который позволяет подгружать русские буквы
\setdefaultlanguage{russian}  % Основной язык документа
\setotherlanguage{english}    % Второстепенный язык документа


%%%%%%%%%% Работа с картинками %%%%%%%%%
\usepackage{graphicx}                  % Для вставки рисунков
\usepackage{graphics}
\graphicspath{{images/}{pictures/}}    % можно указать папки с картинками
\usepackage{wrapfig}                   % Обтекание рисунков и таблиц текстом


%%%%%%%%%% Работа с таблицами %%%%%%%%%%
\usepackage{tabularx}            % новые типы колонок
\usepackage{tabulary}            % и ещё новые типы колонок
\usepackage{array}               % Дополнительная работа с таблицами
\usepackage{longtable}           % Длинные таблицы
\usepackage{multirow}            % Слияние строк в таблице
\usepackage{float}               % возможность позиционировать объекты в нужном месте
\usepackage{booktabs}            % таблицы как в книгах!
\renewcommand{\arraystretch}{1.3} % больше расстояние между строками


% Заповеди из документации к booktabs:
% 1. Будь проще! Глазам должно быть комфортно
% 2. Не используйте вертикальные линни
% 3. Не используйте двойные линии. Как правило, достаточно трёх горизонтальных линий
% 4. Единицы измерения - в шапку таблицы
% 5. Не сокращайте .1 вместо 0.1
% 6. Повторяющееся значение повторяйте, а не говорите "то же"
% 7. Есть сомнения? Выравнивай по левому краю!


%%%%%%%%%% Графика и рисование %%%%%%%%%%
\usepackage{tikz, pgfplots}  % язык для рисования графики из latex'a


%%%%%%%%%% Гиперссылки %%%%%%%%%%
\usepackage{xcolor}              % разные цвета

% Два способа включить в пакете какие-то опции:
%\usepackage[опции]{пакет}
%\usepackage[unicode,colorlinks=true,hyperindex,breaklinks]{hyperref}

\usepackage{hyperref}
\hypersetup{
    unicode=true,           % позволяет использовать юникодные символы
    colorlinks=true,       	% true - цветные ссылки, false - ссылки в рамках
    urlcolor=blue,          % цвет ссылки на url
    linkcolor=red,          % внутренние ссылки
    citecolor=green,        % на библиографию
	pdfnewwindow=true,      % при щелчке в pdf на ссылку откроется новый pdf
	breaklinks              % если ссылка не умещается в одну строку, разбивать ли ее на две части?
}

\usepackage{csquotes}            % Еще инструменты для ссылок


%%%%%%%%%% Другие приятные пакеты %%%%%%%%%
\usepackage{multicol}       % несколько колонок
\usepackage{verbatim}       % для многострочных комментариев

\usepackage{enumitem} % дополнительные плюшки для списков
%  например \begin{enumerate}[resume] позволяет продолжить нумерацию в новом списке

\usepackage{todonotes} % для вставки в документ заметок о том, что осталось сделать
% \todo{Здесь надо коэффициенты исправить}
% \missingfigure{Здесь будет Последний день Помпеи}
% \listoftodos --- печатает все поставленные \todo'шки




\begin{document} % конец преамбулы, начало документа





\begin{minted}[breaklines,linenos]{matlab}
clear
tic

%%%%%%%%%%%%%%%%%%
%%% PARAMETERS %%%
%%%%%%%%%%%%%%%%%%
beta = 0.96; %discount factor
gamma = 2; %inverse of intertemporal elasticity
y = [0.01;1]; %income
Z=[0.9; 1.1]; %TFP shoks
P_y=[0.8 0.2; 0.2 0.8];
P_Z=[0.9 0.1; 0.1 0.9];
bc = 0; %borrowing constraint
delta=1;
alpha=1/3;
N=300;
T=1000;
M=1000;

%%%%%%%%%%
%%% SS %%%
%%%%%%%%%%
P_Z_SS=P_Z^1000; %
Z_SS=P_Z_SS(1,:)*Z;
P_y_SS=P_y^1000; %
h_SS=P_y_SS(1,:)*y;
k_SS = h_SS*((1/beta-1+delta)/alpha/Z_SS)^(1/(alpha-1)); %capital in Ramsey SS

%%%%%%%%%%%%
%%% GRID %%%
%%%%%%%%%%%%
n_a=50;%n el on grid for assets
n_K=50;%n el on grid for capital
a_min = bc;
a_max = 10*k_SS;
K_min=0.8*k_SS;
K_max=2*k_SS;

g = (0:1/(n_a-1):1)';
a = a_min+(a_max-a_min)*g.^3; %grid of wealth
gg=(0:1/(n_K-1):1)';
K = K_min+(K_max-K_min)*gg;

S = [kron(a,ones(4*n_K,1)) repmat(kron(y,ones(2*n_K,1)),n_a,1) repmat(kron(K,ones(2,1)), 2*n_a ,1) repmat(Z,n_a*2*n_K,1)]; %state space
n_s=size(S,1);% number of elements on grid
S_n=[repmat(kron([1;2],ones(2*n_K,1)),n_a,1) repmat([1;2],n_a*2*n_K,1)];

%%%%%%GUESS%%%%
b_ols=[0;1;0;1];
R_2=0;
tfp_n(1,1)=1;
emp_n(:,1)=randi(2,M,1);
cap_n(1,1)=n_K/2;
assets_n(:,1)=randi(n_a,M,1);
state=zeros(M,T);
while R_2<0.96
H=[exp(b_ols(1)) b_ols(2); exp(b_ols(3)) b_ols(4)];
K_f=H(S_n(:,2),1).*S(:,3).^H(S_n(:,2),2);
[~, ind_K]=min(abs(repmat(K_f,1,n_K)-repmat(K',n_s,1)),[],2);
J=[2*ind_K-1 2*ind_K 2*n_K+2*ind_K-1 2*n_K+2*ind_K];
P=zeros(n_s, 4*n_K);
for i=1:n_s
P(i,J(i,1:2))=P_y(S_n(i,1),1)*P_Z(S_n(i,2),:);
P(i,J(i,3:4))=P_y(S_n(i,1),2)*P_Z(S_n(i,2),:);
end

R=1-delta+alpha*S(:,4).*(h_SS./S(:,3)).^(alpha-1);
W=(1-alpha)*S(:,4).*(S(:,3)./h_SS).^alpha;


%%%%%%%%%%%%%%%%%%%%%%%%%%%%%%%%%%
%%% SOLVING FOR VALUE FUNCTION %%%
%%%%%%%%%%%%%%%%%%%%%%%%%%%%%%%%%%
c = repmat(R.*S(:,1)+W.*S(:,2),1,n_a)-repmat(a',n_s,1); 
c = max(c,0); %consumption as function of a,a'
U = (c.^(1-gamma)-1)/(1-gamma); %utility on grid
V = zeros(n_s,1); %initial guess
for t=1:N
V_old = V;
[V, I] = max(U+beta*P*reshape(V,4*n_K,n_a),[],2); %Bellman equation
error = max(abs((V-V_old)./V_old)); 
end
Savings = a(I);


%%%%%%Simulations%%%%%%%%
tfp_n(1,1)=1;
emp_n(1:M/2,1)=1;
emp_n(M/2+1:M,1)=2;
cap_n(1,1)=n_K/2;
assets_n(:,1)=n_a/2*ones(M,1);
state=zeros(M,T);
for t=1:T
state(:,t)=tfp_n(1,t)+2*(cap_n(1,t)-1)+2*n_K*(emp_n(:,t)-1)+4*n_K*(assets_n(:,t)-1); %current state of agents
assets_n(:,t+1)=I(state(:,t)); %evolutionof agents wealth
capital=mean(a(assets_n(:,t+1)));
[~, cap_n(1,t+1)]=min(abs(capital-K'));
x=rand(1,1); %evolution of tfp number
if x<P_Z(tfp_n(1,t),tfp_n(1,t));
tfp_n(1,t+1)=tfp_n(1,t);
else
tfp_n(1,t+1)=3-tfp_n(1,t);
end
x=rand(M,1); %evolution of h number
ind=find(P_y(emp_n(:,t),1)>x);
emp_n(:,t+1)=2;
emp_n(ind,t+1)=1;

end

%%%%%%%%%%%%%%%%%%%
%%%OLS%%%
t=T/10;
b_old=b_ols;
cap=K(cap_n)'; %capital history
Y=log(cap(1,t+1:T+1))';
X(:,1)=2-tfp_n(1,t:T);
X(:,2)=X(:,1).*log(cap(1,t:T))';
X(:,3)=tfp_n(1,t:T)-1;
X(:,4)=X(:,3).*log(cap(1,t:T))';
Y_hat=X*b_ols;
cc=cov(Y,Y_hat);
sst_Y=std(Y);
sst_Y_hat=std(Y_hat);
R_2=cc(1,2)/(sst_Y*sst_Y_hat)
b_ols=(X'*X)\(X'*Y)
b_ols=1/2*b_ols+0.5*b_old;
end
toc
\end{minted}

\end{document} % конец документа
