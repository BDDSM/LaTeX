
%!TEX TS-program = xelatex
\documentclass[12pt, a4paper]{article}  

%%%%%%%%%% Математика %%%%%%%%%%
\usepackage{amsmath,amsfonts,amssymb,amsthm,mathtools} 
%\mathtoolsset{showonlyrefs=true}  % Показывать номера только у тех формул, на которые есть \eqref{} в тексте.
%\usepackage{leqno} % Нумерация формул слева


%%%%%%%%%%%%%%% Шрифты %%%%%%%%%%%
\usepackage{fontspec}         % пакет для подгрузки шрифтов
\setmainfont{Phorssa}   % задаёт основной шрифт документа

\defaultfontfeatures{Mapping=tex-text}

% why do we need \newfontfamily:
% http://tex.stackexchange.com/questions/91507/
\newfontfamily{\cyrillicfonttt}{Phorssa}
\newfontfamily{\cyrillicfont}{Phorssa}
\newfontfamily{\cyrillicfontsf}{Phorssa}
% Грубо говоря иногда polyglossia начинает балбесничать и не видит структуры кириллических шрифтов. Эти трое бравых парней спасают ситуацию и доопределяют те куски, которые Тех не увидел.

\usepackage{unicode-math}     % пакет для установки математического шрифта
\setmathfont{Phorssa}      % шрифт для математики
% \setmathfont[math-style=ISO]{Asana Math}
% Можно делать смену начертания с помощью разных стилей

% Конкретный символ из конкретного шрифта
% \setmathfont[range=\int]{Neo Euler}


\usepackage{polyglossia}      % Пакет, который позволяет подгружать русские буквы
\setdefaultlanguage{russian}  % Основной язык документа
\setotherlanguage{english}    % Второстепенный язык документа


% Заголовок
%\author{Уютный факультатив} 
%\title{Xe\LaTeX. Работа со шрифтами.}
%\date{\today}


\begin{document}
	
\setcounter{section}{2}
	\section{Угроза Тишину Александру}
	
Александр! Если вы и дальше будете придумывать такие сложные домашки, то мне придется забрать у вас самое дорогое - ваши баллы за дипломную работу. Будем считать, что стандартное домшнее задание можно делать не более двух часов. Если вы будете жестить, придумывая сложные д/з, то я буду отрезать по три балла от вашей итоговой оценки за каждый дополнительный час, потраченный на \LaTeX, и присылать их вам в конверте.
 
Формула усечения имеет вид: $$f=100-3(t-2),$$ где $f$ (от слова fatal) - итоговый балл за ВКР, $t\ge2$ - время, затраченное мною на выполнение ваших домашек (в часах).

Считаю излишним напоминать, что обращаться в полицию бессмысленно.

С уважением, ваш научрук.

Ха-ха-ха!

	
	
\end{document}