%!TEX TS-program = xelatex
\documentclass[12pt, a4paper, oneside]{article}

\usepackage{amsmath,amsfonts,amssymb,amsthm,mathtools}  % пакеты для математики

\usepackage[utf8]{inputenc} % задание utf8 кодировки исходного tex файла
\usepackage[british,russian]{babel} % выбор языка для документа

\usepackage{fontspec}         % пакет для подгрузки шрифтов
\setmainfont{Helvetica}   % задаёт основной шрифт документа

% why do we need \newfontfamily:
% http://tex.stackexchange.com/questions/91507/
\newfontfamily{\cyrillicfonttt}{Helvetica}
\newfontfamily{\cyrillicfont}{Helvetica}
\newfontfamily{\cyrillicfontsf}{Helvetica}

\usepackage{unicode-math}     % пакет для установки математического шрифта
\setmathfont{Neo Euler}      % шрифт для математики
% \setmathfont[math-style=ISO]{Asana Math}
% Можно делать смену начертания с помощью разных стилей

% Конкретный символ из конкретного шрифта
% \setmathfont[range=\int]{Neo Euler}

%%%%%%%%%% Работа с картинками %%%%%%%%%
\usepackage{graphicx}                  % Для вставки рисунков
\usepackage{graphics}
\graphicspath{{images/}{pictures/}}    % можно указать папки с картинками
\usepackage{wrapfig}                   % Обтекание рисунков и таблиц текстом

%%%%%%%%%%%%%%%%%%%%%%%% Графики и рисование %%%%%%%%%%%%%%%%%%%%%%%%%%%%%%%%%
\usepackage{tikz, pgfplots}  % язык для рисования графики из latex'a

%%%%%%%%%% Гиперссылки %%%%%%%%%%
\usepackage{xcolor}              % разные цвета

\usepackage{hyperref}
\hypersetup{
	unicode=true,           % позволяет использовать юникодные символы
	colorlinks=true,       	% true - цветные ссылки, false - ссылки в рамках
	urlcolor=blue,          % цвет ссылки на url
	linkcolor=red,          % внутренние ссылки
	citecolor=green,        % на библиографию
	pdfnewwindow=true,      % при щелчке в pdf на ссылку откроется новый pdf
	breaklinks              % если ссылка не умещается в одну строку, разбивать ли ее на две части?
}


\usepackage{todonotes} % для вставки в документ заметок о том, что осталось сделать
% \todo{Здесь надо коэффициенты исправить}
% \missingfigure{Здесь будет Последний день Помпеи}
% \listoftodos --- печатает все поставленные \todo'шки

\usepackage{enumitem} % дополнительные плюшки для списков
%  например \begin{enumerate}[resume] позволяет продолжить нумерацию в новом списке

\usepackage[paper=a4paper, top=20mm, bottom=15mm,left=20mm,right=15mm]{geometry}
\usepackage{indentfirst}       % установка отступа в первом абзаце главы

\usepackage{setspace}
\setstretch{1.15}  % Межстрочный интервал
\setlength{\parskip}{4mm}   % Расстояние между абзацами
% Разные длины в латехе https://en.wikibooks.org/wiki/LaTeX/Lengths


\usepackage{xcolor} % Enabling mixing colors and color's call by 'svgnames'

\definecolor{MyColor1}{rgb}{0.2,0.4,0.6} %mix personal color
\newcommand{\textb}{\color{Black} \usefont{OT1}{lmss}{m}{n}}
\newcommand{\blue}{\color{MyColor1} \usefont{OT1}{lmss}{m}{n}}
\newcommand{\blueb}{\color{MyColor1} \usefont{OT1}{lmss}{b}{n}}
\newcommand{\red}{\color{LightCoral} \usefont{OT1}{lmss}{m}{n}}
\newcommand{\green}{\color{Turquoise} \usefont{OT1}{lmss}{m}{n}}

\usepackage{titlesec}
\usepackage{sectsty}
%%%%%%%%%%%%%%%%%%%%%%%%
%set section/subsections HEADINGS font and color
\sectionfont{\color{MyColor1}}  % sets colour of sections
\subsectionfont{\color{MyColor1}}  % sets colour of sections

%set section enumerator to arabic number (see footnotes markings alternatives)
\renewcommand\thesection{\arabic{section}.} %define sections numbering
\renewcommand\thesubsection{\thesection\arabic{subsection}} %subsec.num.

%define new section style
\newcommand{\mysection}{
	\titleformat{\section} [runin] {\usefont{OT1}{lmss}{b}{n}\color{MyColor1}} 
	{\thesection} {3pt} {} } 


%	CAPTIONS
\usepackage{caption}
\usepackage{subcaption}
%%%%%%%%%%%%%%%%%%%%%%%%
\captionsetup[figure]{labelfont={color=Turquoise}}

\usepackage[normalem]{ulem}  % для зачекивания текста

\pagestyle{empty}

\begin{document}

\section*{Задание 6 (20 баллов)  }

Не забывай, где находится  \href{https://fulyankin.github.io/LaTeX/}{страничку курса} с кучей шпаргалок! А также где лежит \href{https://docs.google.com/forms/d/e/1FAIpQLSe11kxKVfv07iCL1E9yNX7ll9swKImiVwRr1H70lslGzInRSg/viewform}{уютная гугл-форма.} Не стесняйтесь просить о помощи, если она вам необходима. \textbf{Все баллы в дз дополнительные. Дедлайн по этому заданию:  }  

\todo[inline]{Поставить ссылку на базу задач}

\subsection*{[20]   Упражнение 1 (Правильная эконометрика (или мёд)) }

\todo[inline]{Поставить ссылку на статью Анатольева}

Ещё разок прочитайте статью Анатольева о правильном оформлении эконометрических расчётов. Никогда не вставляйте в свою работу протоколы из статы или ивзглядов (это шутка про Eviews). Оформляйте всё по-человечески!  И вообще, делайте эконометрику в R!

Винни-Пух работает риск-менеджером в очень серьёзном банке. Ровно в 16:15 на столе у главы правления банка, Пяточка, должен лежать отчёт, который содержит в себе величину показателя VaR и некоторую другую статистику.

\textbf{VaR (Value at Risk)}  --- это один из способов измерить риск. \textbf{VaR} --- это величина убытков, которая с заданной вероятностью (обычно 95\% или 99\%) не будет превышена (да-да, это просто квантиль). Иначе говоря, каждый раз, когда начальник берёт отчёт, он видит, что отдел риск-менеджеров на 95\% уверен, что убытки в следующем периоде не превысят величину VaR.

Винни-Пух приходит на работу к 9 часам утра. У него есть код для расчётов. Винни проделывает в R расчёты и копирует все цифры в Word. На это уходит несколько часов. Оставшееся время до 16:15 Винни курит, пьёт кофе и перетирает со своим братаном из отдела алгоритмического трейдинга --- Кроликом.

Недавно Винни-Пух узнал о том, что если присоединить к R какой-то \LaTeX{}, то можно будет тупо жать каждое утро одну кнопку, а остальное время протирать штаны. Помогите Винни-Пуху протирать штаны более эффективно. Напишите скрипт в связке R+\LaTeX{}, который будет проделывать все расчёты по подгружаемому csv-файлу и делать красивую pdf-ку со всей необходимой начальнику информацией. Все скрипты Винни-Пуха лежат в архиве.

\todo[inline]{Поставить ссылку на архив}

\begin{itemize}
	\item[$(18)$] Автоматизируйте создание отчёта Винни-Пухом. В отчёте вообще не должно быть кода! Это документ!!!
	\item[$(2)$]  В каком Банке работает Винни-Пух? Нет, это не сбербанк, а я не Винни-Пух. Сейчас вот обидно было...
\end{itemize}


\subsection*{[Бесценно]  Упражнение 2  }

Вспомните как в R работают циклы и условия. В следующий раз нам это очень сильно будет нужно!

\end{document}

