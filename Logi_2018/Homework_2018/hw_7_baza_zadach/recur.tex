%!TEX TS-program = xelatex
\documentclass[12pt, a4paper]{article}

% Этот шаблон документа разработан в 2014 году
% Данилом Фёдоровых (danil@fedorovykh.ru) 
% для использования в курсе 
% <<Документы и презентации в \LaTeX>>, записанном НИУ ВШЭ
% для Coursera.org: http://coursera.org/course/latex .
% Исходная версия шаблона --- 
% https://www.writelatex.com/coursera/latex/5.3

% В этом документе преамбула

%%% Работа с русским языком
\usepackage{cmap}					% поиск в PDF
\usepackage{mathtext} 				% русские буквы в формулах
\usepackage[T2A]{fontenc}			% кодировка
\usepackage[utf8]{inputenc}			% кодировка исходного текста
\usepackage[english,russian]{babel}	% локализация и переносы
\usepackage{indentfirst}
\frenchspacing

\renewcommand{\epsilon}{\ensuremath{\varepsilon}}
\renewcommand{\phi}{\ensuremath{\varphi}}
\renewcommand{\kappa}{\ensuremath{\varkappa}}
\renewcommand{\le}{\ensuremath{\leqslant}}
\renewcommand{\leq}{\ensuremath{\leqslant}}
\renewcommand{\ge}{\ensuremath{\geqslant}}
\renewcommand{\geq}{\ensuremath{\geqslant}}
\renewcommand{\emptyset}{\varnothing}

%%% Дополнительная работа с математикой
\usepackage{amsmath,amsfonts,amssymb,amsthm,mathtools} % AMS
\usepackage{icomma} % "Умная" запятая: $0,2$ --- число, $0, 2$ --- перечисление

%% Номера формул
%\mathtoolsset{showonlyrefs=true} % Показывать номера только у тех формул, на которые есть \eqref{} в тексте.
%\usepackage{leqno} % Нумереация формул слева

%% Свои команды
\DeclareMathOperator{\sgn}{\mathop{sgn}}

%% Перенос знаков в формулах (по Львовскому)
\newcommand*{\hm}[1]{#1\nobreak\discretionary{}
{\hbox{$\mathsurround=0pt #1$}}{}}

%%% Работа с картинками
\usepackage{graphicx}  % Для вставки рисунков
\graphicspath{{images/}{images2/}}  % папки с картинками
\setlength\fboxsep{3pt} % Отступ рамки \fbox{} от рисунка
\setlength\fboxrule{1pt} % Толщина линий рамки \fbox{}
\usepackage{wrapfig} % Обтекание рисунков текстом

%%% Работа с таблицами
\usepackage{array,tabularx,tabulary,booktabs} % Дополнительная работа с таблицами
\usepackage{longtable}  % Длинные таблицы
\usepackage{multirow} % Слияние строк в таблице

%%% Теоремы
\theoremstyle{plain} % Это стиль по умолчанию, его можно не переопределять.
\newtheorem{theorem}{Теорема}[section]
\newtheorem{proposition}[theorem]{Утверждение}
 
\theoremstyle{definition} % "Определение"
\newtheorem{corollary}{Следствие}[theorem]
\newtheorem{problem}{Задача}[section]
 
\theoremstyle{remark} % "Примечание"
\newtheorem*{nonum}{Решение}

%%% Программирование
\usepackage{etoolbox} % логические операторы

%%% Страница
\usepackage{extsizes} % Возможность сделать 14-й шрифт
\usepackage{geometry} % Простой способ задавать поля
	\geometry{top=25mm}
	\geometry{bottom=35mm}
	\geometry{left=35mm}
	\geometry{right=20mm}
 %
%\usepackage{fancyhdr} % Колонтитулы
% 	\pagestyle{fancy}
 	%\renewcommand{\headrulewidth}{0pt}  % Толщина линейки, отчеркивающей верхний колонтитул
% 	\lfoot{Нижний левый}
% 	\rfoot{Нижний правый}
% 	\rhead{Верхний правый}
% 	\chead{Верхний в центре}
% 	\lhead{Верхний левый}
%	\cfoot{Нижний в центре} % По умолчанию здесь номер страницы

\usepackage{setspace} % Интерлиньяж
%\onehalfspacing % Интерлиньяж 1.5
%\doublespacing % Интерлиньяж 2
%\singlespacing % Интерлиньяж 1

\usepackage{lastpage} % Узнать, сколько всего страниц в документе.

\usepackage{soul} % Модификаторы начертания

\usepackage{hyperref}
\usepackage[usenames,dvipsnames,svgnames,table,rgb]{xcolor}
\hypersetup{				% Гиперссылки
    unicode=true,           % русские буквы в раздела PDF
    pdftitle={Заголовок},   % Заголовок
    pdfauthor={Автор},      % Автор
    pdfsubject={Тема},      % Тема
    pdfcreator={Создатель}, % Создатель
    pdfproducer={Производитель}, % Производитель
    pdfkeywords={keyword1} {key2} {key3}, % Ключевые слова
    colorlinks=true,       	% false: ссылки в рамках; true: цветные ссылки
    linkcolor=red,          % внутренние ссылки
    citecolor=black,        % на библиографию
    filecolor=magenta,      % на файлы
    urlcolor=cyan           % на URL
}

\usepackage{csquotes} % Еще инструменты для ссылок

%\usepackage[style=authoryear,maxcitenames=2,backend=biber,sorting=nty]{biblatex}

\usepackage{multicol} % Несколько колонок

\usepackage{tikz} % Работа с графикой
\usepackage{pgfplots}
\usepackage{pgfplotstable}




% Специальный пакет для оформления задач! 
\newtheorem{problem}{Задача}

\usepackage{answers}
\Newassociation{sol}{solution}{solution_file}
% sol --- имя окружения внутри задач
% solution --- имя окружения внутри solution_file
% solution_file --- имя файла в который будет идти запись решений


\begin{document}

% Открываем файл, куда будут записываться решения. 
\Opensolutionfile{solution_file}[all_solutions]


\begin{problem}
Решить следующие однородное рекурентное уравнение:
\[y_{n+2} = 5 y_{n+1} - 6 y_n, \text{ если } y_0 = 0, y_1 = 1\]
\begin{sol}

\end{sol}
\end{problem}



\begin{problem}
Решить следующие однородное рекурентное уравнение:
\[y_{n+2} = 4 y_{n+1} - 4 y_n, \text{ если } y_1 = 2, y_2 = 4\] 
\begin{sol}

\end{sol}
\end{problem}



\begin{problem}
Решить следующие однородное рекурентное уравнение:
\[2y_{n+2} = 5 y_{n+1} - 3 y_n, \text{ если } y_0 = 0, y_1 = 1\]
\begin{sol}

\end{sol}
\end{problem}



\begin{problem}
Решить следующие неоднородное рекурентное уравнение:
\[2y_{n+2} - 5 y_{n+1} + 3 y_n = 2 + 3n, \text{ если } y_0 = 0, y_1 = 1\]
\begin{sol}

\end{sol}
\end{problem}



\begin{problem}
Решить следующие неоднородное рекурентное уравнение:
\[y_{n+2} - 5 y_{n+1} + 6 y_n = 2^n, \text{ если } y_0 = 0, y_1 = 1\]
\begin{sol}

\end{sol}
\end{problem}



\begin{problem}
Решить следующие неоднородное рекурентное уравнение:
\[y_{n+2} - 5 y_{n+1} + 6 y_n = 2 + 3n, \text{ если } y_0 = 0, y_1 = 1\]
\begin{sol}

\end{sol}
\end{problem}



\begin{problem}
Решить следующие неоднородное рекурентное уравнение:
\[y_{n+2} - 4 y_{n+1} + 4 y_n = 2^n, \text{ если } y_1 = 2, y_2 = 4\] 
\begin{sol}

\end{sol}
\end{problem}



\begin{problem}
Решить следующее рекурентное уравнение:
\[y_{n+4} = 5 y_{n+3} - 6 y_{n+2} - 4 y_{n+1} + 8 y_n\]
\begin{sol}

\end{sol}
\end{problem}


\begin{problem}
Решить следующее рекурентное уравнение:
\[y_{n+2} = 2 y_{n+1} - 2 y_n\]
\begin{sol}

\end{sol}
\end{problem}


\begin{problem}
Решить следующее рекурентное уравнение:
\[y_{n+2} + 9 y_{n} = 0\]
\begin{sol}

\end{sol}
\end{problem}



\begin{problem}
Решить следующее рекурентное уравнение:
\[y_{n+2} - 2 y_{n+1} \cos{\varphi} + y_n = 0, \text{ если } y_1 = \cos{\varphi}, y_2 = \cos{2\varphi}\]
\begin{sol}

\end{sol}
\end{problem}



\begin{problem}
Решить следующее рекурентное уравнение:
\[y_{n+k} - C_k^1 y_{n+k-1} + C_k^2 y_{n+k-2} - \ldots + (-1)^k C_k^k y_k = 0\] 
Будет ли последовательность $y_n = n^t$ решением этого рекурентного соотношения?
\begin{sol}

\end{sol}
\end{problem}



\begin{problem}
Пара кроликов раз в месяц приносит приплод из двух крольчат (самец и самка), при этом новорождённые крольчата через два месяца после рождения уже приносят приплод. Сколько пар кроликов появится через год, если в начале года была одна новорождённая пара кроликов. В течение года кролики не погибали.
\begin{sol}

\end{sol}
\end{problem}



\begin{problem}
На доску выписаны числа $a_1, a_2, a_3, \ldots, a_200$. Известно, что $a_1 = 3$, $a_2 = 9$. Найдите $a_{200}$, если для любого натурального числа $n$ справедливо равенство $a_{n+2} = a_{n+1} - a_n$.
\begin{sol}

\end{sol}
\end{problem}



% Закрываем файл, куда мы записывали решения и вставляем его в конце списка задач. 
\Closesolutionfile{solution_file}

% Вставляем решения. Можно их не вставлять или настроить пакет так, чтобы они шли непосредственно после каждой задачи.
% \begin{solution}{1}
\end{solution}
\begin{solution}{3}
\end{solution}
\begin{solution}{4}
\end{solution}



\end{document}
