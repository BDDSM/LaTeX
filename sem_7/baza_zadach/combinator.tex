%!TEX TS-program = xelatex
\documentclass[12pt, a4paper]{article}

% Этот шаблон документа разработан в 2014 году
% Данилом Фёдоровых (danil@fedorovykh.ru) 
% для использования в курсе 
% <<Документы и презентации в \LaTeX>>, записанном НИУ ВШЭ
% для Coursera.org: http://coursera.org/course/latex .
% Исходная версия шаблона --- 
% https://www.writelatex.com/coursera/latex/5.3

% В этом документе преамбула

%%% Работа с русским языком
\usepackage{cmap}					% поиск в PDF
\usepackage{mathtext} 				% русские буквы в формулах
\usepackage[T2A]{fontenc}			% кодировка
\usepackage[utf8]{inputenc}			% кодировка исходного текста
\usepackage[english,russian]{babel}	% локализация и переносы
\usepackage{indentfirst}
\frenchspacing

\renewcommand{\epsilon}{\ensuremath{\varepsilon}}
\renewcommand{\phi}{\ensuremath{\varphi}}
\renewcommand{\kappa}{\ensuremath{\varkappa}}
\renewcommand{\le}{\ensuremath{\leqslant}}
\renewcommand{\leq}{\ensuremath{\leqslant}}
\renewcommand{\ge}{\ensuremath{\geqslant}}
\renewcommand{\geq}{\ensuremath{\geqslant}}
\renewcommand{\emptyset}{\varnothing}

%%% Дополнительная работа с математикой
\usepackage{amsmath,amsfonts,amssymb,amsthm,mathtools} % AMS
\usepackage{icomma} % "Умная" запятая: $0,2$ --- число, $0, 2$ --- перечисление

%% Номера формул
%\mathtoolsset{showonlyrefs=true} % Показывать номера только у тех формул, на которые есть \eqref{} в тексте.
%\usepackage{leqno} % Нумереация формул слева

%% Свои команды
\DeclareMathOperator{\sgn}{\mathop{sgn}}

%% Перенос знаков в формулах (по Львовскому)
\newcommand*{\hm}[1]{#1\nobreak\discretionary{}
{\hbox{$\mathsurround=0pt #1$}}{}}

%%% Работа с картинками
\usepackage{graphicx}  % Для вставки рисунков
\graphicspath{{images/}{images2/}}  % папки с картинками
\setlength\fboxsep{3pt} % Отступ рамки \fbox{} от рисунка
\setlength\fboxrule{1pt} % Толщина линий рамки \fbox{}
\usepackage{wrapfig} % Обтекание рисунков текстом

%%% Работа с таблицами
\usepackage{array,tabularx,tabulary,booktabs} % Дополнительная работа с таблицами
\usepackage{longtable}  % Длинные таблицы
\usepackage{multirow} % Слияние строк в таблице

%%% Теоремы
\theoremstyle{plain} % Это стиль по умолчанию, его можно не переопределять.
\newtheorem{theorem}{Теорема}[section]
\newtheorem{proposition}[theorem]{Утверждение}
 
\theoremstyle{definition} % "Определение"
\newtheorem{corollary}{Следствие}[theorem]
\newtheorem{problem}{Задача}[section]
 
\theoremstyle{remark} % "Примечание"
\newtheorem*{nonum}{Решение}

%%% Программирование
\usepackage{etoolbox} % логические операторы

%%% Страница
\usepackage{extsizes} % Возможность сделать 14-й шрифт
\usepackage{geometry} % Простой способ задавать поля
	\geometry{top=25mm}
	\geometry{bottom=35mm}
	\geometry{left=35mm}
	\geometry{right=20mm}
 %
%\usepackage{fancyhdr} % Колонтитулы
% 	\pagestyle{fancy}
 	%\renewcommand{\headrulewidth}{0pt}  % Толщина линейки, отчеркивающей верхний колонтитул
% 	\lfoot{Нижний левый}
% 	\rfoot{Нижний правый}
% 	\rhead{Верхний правый}
% 	\chead{Верхний в центре}
% 	\lhead{Верхний левый}
%	\cfoot{Нижний в центре} % По умолчанию здесь номер страницы

\usepackage{setspace} % Интерлиньяж
%\onehalfspacing % Интерлиньяж 1.5
%\doublespacing % Интерлиньяж 2
%\singlespacing % Интерлиньяж 1

\usepackage{lastpage} % Узнать, сколько всего страниц в документе.

\usepackage{soul} % Модификаторы начертания

\usepackage{hyperref}
\usepackage[usenames,dvipsnames,svgnames,table,rgb]{xcolor}
\hypersetup{				% Гиперссылки
    unicode=true,           % русские буквы в раздела PDF
    pdftitle={Заголовок},   % Заголовок
    pdfauthor={Автор},      % Автор
    pdfsubject={Тема},      % Тема
    pdfcreator={Создатель}, % Создатель
    pdfproducer={Производитель}, % Производитель
    pdfkeywords={keyword1} {key2} {key3}, % Ключевые слова
    colorlinks=true,       	% false: ссылки в рамках; true: цветные ссылки
    linkcolor=red,          % внутренние ссылки
    citecolor=black,        % на библиографию
    filecolor=magenta,      % на файлы
    urlcolor=cyan           % на URL
}

\usepackage{csquotes} % Еще инструменты для ссылок

%\usepackage[style=authoryear,maxcitenames=2,backend=biber,sorting=nty]{biblatex}

\usepackage{multicol} % Несколько колонок

\usepackage{tikz} % Работа с графикой
\usepackage{pgfplots}
\usepackage{pgfplotstable}




% Специальный пакет для оформления задач! 
\newtheorem{problem}{Задача}

\usepackage{answers}
\Newassociation{sol}{solution}{solution_file}
% sol --- имя окружения внутри задач
% solution --- имя окружения внутри solution_file
% solution_file --- имя файла в который будет идти запись решений


\begin{document}

% Открываем файл, куда будут записываться решения. 
\Opensolutionfile{solution_file}[all_solutions]


\begin{problem}
В корзине лежат 12 яблок и 10 апельсинов. Ваня выбирает из неё ябллоко или апельсин, после чего Надя берёт и яблоко, и апельсин. В каком случае Надя имеет большую свободу выбора: если Ваня взял яблоко или если он взял апельсин? 
\begin{sol}

\end{sol}
\end{problem}



\begin{problem}
Тридцать первокурсников с эконома попали на необитаемый остров. Необходимо выбрать вождя, его заместителя и пять человек, которые будут сварены для обеда. Сколько существует способов сделать это?
\begin{sol}

\end{sol}
\end{problem}



\begin{problem}
В депо есть куча вагонов 8 разных типов. Нужно составит поезд из 6 вагонов. Сколько существует способов составить поезд из 6 вагонов разных типов? Из 6 ванов любых типов? Из 6 ванонов любых типов, но так чтобы был вагон-ресторан.
\begin{sol}

\end{sol}
\end{problem}



\begin{problem}
В лаборатории есть 10 разных мышей. Сколько есть способов выбрать 5 мышей для экспперимента? 
\begin{sol}

\end{sol}
\end{problem}



\begin{problem}
Двое друзей коллекционируют монеты. У первого 9 монет. У второго 7 монет. Все монеты разные. Сколько существует способов обменять две монеты одного друга на две монеты другого?
\begin{sol}

\end{sol}
\end{problem}



\begin{problem}
В кучке лежит 28 костяшек домино. Сколько способов выдать игроку 7 костяше так, чтобы среди них оказался хотя бы один дубль?
\begin{sol}

\end{sol}
\end{problem}



\begin{problem}
Сколькими способами можно составить расписание авиарейсов, если их всего 7, а в день делается от 2 до 4 перелётов?
\begin{sol}

\end{sol}
\end{problem}



\begin{problem}
Маша и Таня хотят в столовой сидеть рядом. Коля не хочет сидеть последним. Сколько существует способов сесть, если всего в столовой 4 человека? А если 8 человек? 
\begin{sol}

\end{sol}
\end{problem}



\begin{problem}
Жених и невеста выбирают трёх-ярусный свадебный торт. На выбор есть 5 типов ярусов. Сколько разных тортов может предложить кондитер, если бисквитных ярусов в торте может быть от 0 до 2, а остальных по одному. 
\begin{sol}

\end{sol}
\end{problem}



\begin{problem}
В тренировочной группе 5 новичков и 3 опытных альпиниста. В поход идёт 6 человек, среди которых от 2 опытных альпинистов. Сколько существует способов сформировать походную группу?
\begin{sol}

\end{sol}
\end{problem}



\begin{problem}
20 перваков и куратор пошли в кино и хотят сидеть в одном ряду. Порядок рассадки важен, так как все перваки разные. Куратор хочет сидеть скраю. Коля, Вася и Петя не могут сидеть втроём, так как сразу же начинают спорить о коинтеграции и фильм смотреть невозможно! Сколько существует способов сесть?
\begin{sol}

\end{sol}
\end{problem}



\begin{problem}
На полке стоит 40 книг. Среди них есть три томика Пушкина. Сколько существует способов расставить книги так, чтобы три тома пушкина располагались по возрастанию? Чтобы никакие два тома не примыкали друг к другу?
\begin{sol}

\end{sol}
\end{problem}



\begin{problem}
При начале деловой встричи каждый из 6 партнёров пожал другому руку. По окончанию встречи каждый дал каждому свою визитную карточку. Сколько было сделано рукопожатий? Сколько визитных карточек перешло из рук в руки?
\begin{sol}

\end{sol}
\end{problem}



\begin{problem}
В кассу стоят 9 человек (3 мужчины, 4 женщины и два ребёнка). Сколько существует способов поставить между двумя некоторыми мужчинами двое детей и одну женщину?
\begin{sol}

\end{sol}
\end{problem}


\begin{problem}
Колода из 36 карт. Сколько существует комбинаций раскладки карт таких, что места расположения тузов образуют арифметическую прогрессию с шагом равным семи?
\begin{sol}

\end{sol}
\end{problem}



\begin{problem}
В кондитерском магазине продавались пирожные 4 видов: корзиночки, наполеоны, песочные и эклеры. Сколькими способами можно купить 7 пирожных?
\begin{sol}

\end{sol}
\end{problem}


\begin{problem}
В продаже имеются розы трёх разных цветов. Сколькими способами можно составить букет из 7 роз? 
\begin{sol}

\end{sol}
\end{problem}


\begin{problem}
12 зеркал нужно развесить по 8 залам. Сколько существует способов сделать это, если в каждом из залов должно висеть хотя бы одно зеркало? 
\begin{sol}

\end{sol}
\end{problem}



\begin{problem}
Один человек имеет шесть друзей и в течение 20 дней приглашает к себе в гости троих из них, так чтобы компания ни разу не повторилась. Сколькими способами он может это сделать? (вспомни упражнение про дежурства!)
\begin{sol}

\end{sol}
\end{problem}



\begin{problem}
Из группы, состоящей из семи юношей и четырёх девушек надо выбрать шесть человек так, чтобы среди них было не менее двух девушек. Сколькими способами это можно сделать? 
\begin{sol}

\end{sol}
\end{problem}



\begin{problem}
В вазе стоят 10 красных и 4 розовых гвоздики. Сколькими способами можно выбрать три цветка из вазы?
\begin{sol}

\end{sol}
\end{problem}



\begin{problem}
Четверо студентов сдают экзамен. Сколькими способами им могут быть выставлены оценки, если известно, что никто из них не получит неудовлетворительную оценку? 
\begin{sol}

\end{sol}
\end{problem}



\begin{problem}
Садовник в течение трёх дней должен посадить 10 деревьев. Сколькими способами он может распределить работу по дням, если будет сажать не менее одного дерева в день? 
\begin{sol}

\end{sol}
\end{problem}



\begin{problem}
Семь яблок и три апельсина надо положить в два пакета так, чтобы в каждом пакете был хотя бы один апельсин и чтобы количество фруктов в них было одинаково. Сколькими способами это можно сделать? 
\begin{sol}

\end{sol}
\end{problem}



% Закрываем файл, куда мы записывали решения и вставляем его в конце списка задач. 
\Closesolutionfile{solution_file}

% Вставляем решения. Можно их не вставлять или настроить пакет так, чтобы они шли непосредственно после каждой задачи.
% \begin{solution}{1}
\end{solution}
\begin{solution}{3}
\end{solution}
\begin{solution}{4}
\end{solution}



\end{document}
