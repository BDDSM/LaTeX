\documentclass[12pt, a4paper]{article}


\usepackage{etex}     
% расширение классического tex в частности позволяет подгружать гораздо больше пакетов, чем мы и займёмся далее


%%%%%%%%%% Русский язык %%%%%%%%%%
\usepackage[british,russian]{babel} % выбор языка для документа
\usepackage[utf8]{inputenc}         % задание utf8 кодировки исходного tex файла
\usepackage[X2,T2A]{fontenc}        % кодировка
\usepackage{cmap}					% поиск в PDF


%%%%%%%%%% Математика %%%%%%%%%%
\usepackage{amsmath,amsfonts,amssymb,amsthm,mathtools} 

%\mathtoolsset{showonlyrefs=true}  % Показывать номера только у тех формул, на которые есть \eqref{} в тексте.
%\usepackage{leqno} % Нумерация формул слева


%%%%%%%%%% Работа с картинками %%%%%%%%%
\usepackage{graphicx}                  % Для вставки рисунков
\usepackage{graphics} 
\graphicspath{{images/}{pictures/}}    % можно указать папки с картинками
\usepackage{wrapfig}                   % Обтекание рисунков и таблиц текстом
\usepackage{subfigure}                 % для создания нескольких рисунков внутри одного


%%%%%%%%%% Работа с таблицами %%%%%%%%%%
\usepackage{tabularx}            % новые типы колонок
\usepackage{tabulary}            % и ещё новые типы колонок
\usepackage{array}               % Дополнительная работа с таблицами
\usepackage{longtable}           % Длинные таблицы
\usepackage{multirow}            % Слияние строк в таблице
\usepackage{float}               % возможность позиционировать объекты в нужном месте 
\usepackage{booktabs}            % таблицы как в книгах!  
\renewcommand{\arraystretch}{1.3} % больше расстояние между строками



%%%%%%%%%% Графика и рисование %%%%%%%%%%
\usepackage{tikz, pgfplots}  % язык для рисования графики из latex'a
\usepackage{amscd}                  %Пакеты для рисования 
\usepackage[matrix,arrow,curve]{xy} %комунитативных диаграмм




\title{Графика в LaTeX, TikZ}
\date{\today}

\begin{document} % конец преамбулы, начало документа

\section{Внутренние средства для графики в \LaTeX{} или как не надо делать!}


\begin{center}
\begin{picture}(200,100)
\put(50,50){\oval(50,100)}
\put(150,50){\oval(50,100)}
\put(50,80){\circle*{2}}
\put(43,80){\footnotesize{1}}
\put(50,60){\circle*{2}}
\put(43,60){\footnotesize{2}}
\put(50,40){\circle*{2}}
\put(43,40){\footnotesize{3}}
\put(50,20){\circle*{2}}
\put(43,20){\footnotesize{4}}
\put(150,80){\circle*{2}}
\put(153,80){\footnotesize{5}}
\put(150,60){\circle*{2}}
\put(153,60){\footnotesize{11}}
\put(150,40){\circle*{2}}
\put(153,40){\footnotesize{12}}
\put(150,20){\circle*{2}}
\put(153,20){\footnotesize{17}}
\put(53,80){\vector(1,0){94}}
\put(147,80){\vector(-1,0){94}}
\put(53,60){\vector(1,0){94}}
\put(147,60){\vector(-1,0){94}}
\put(53,40){\vector(1,0){94}}
\put(147,40){\vector(-1,0){94}}
\put(53,20){\vector(1,0){94}}
\put(147,20){\vector(-1,0){94}}
\end{picture}
\end{center}

\begin{figure}[h!]
\begin{center}
\[ \xymatrix{
(1,1) \ar[r] &  (1,2) \ar[ld]  &  (1,3) \ar[r]  & (1,4) \ar[ld]  & (1,5)  \ar[r] & \dots  \\
(2,1) \ar[d] & (2,2) \ar[ru]  & (2,3) \ar[ld]  & (2,4) \ar[ur] & (2,5) &  \dots\\
(3,1) \ar[ru] & (3,2) \ar[ld] & (3,3) \ar[ur] & (3,4) & (3,5)  & \dots \\
(4,1) \ar[d] & (4,2) \ar[ur] & (4,3) & (4,4) & (4,5)  & \dots \\
(5,1) \ar[ur] & (5,2) & (5,3) & (5,4) & (5,5)  & \dots \\
\dots & \dots & \dots & \dots & \dots  }
\]
\caption{Нумерация элементов в множестве пар натуральных чисел $\mathbb{N}^2$}\label{n^2}
\end{center}
\end{figure}



\section{Великий и могучий Tikz}
\subsection{ Основные команды} 


% Никогда не забывать в конце каждой строки ставить ;
% Tikz очень капризен по отношению к этому!

\begin{tikzpicture}
\draw[green, dashed] (0,1) -- (2,1) (0,2) -- (2,2);
\draw[->] (0,1) -- (-1,1);
\draw[blue,fill=yellow](2,2) circle [radius = 0.5];
\end{tikzpicture}


\begin{tikzpicture}
\foreach \x in {0,...,9} 
  \draw (\x,0) circle (0.4);
\end{tikzpicture}

% \draw [domain=<xmin>:<xmax>] plot (\x, {function})

\begin{tikzpicture}
\draw [help lines] (-2,0) grid (2,4); 
\draw [->] (-2.2,0) -- (2.2,0); 
\draw [->] (0,0) -- (0,4.2); 
\draw [green, thick, domain=-2:2] plot (\x, {4-\x*\x}); 
\draw [domain=-2:2, samples=50] plot (\x, {1+cos(pi*\x r});
\end{tikzpicture}

\subsection{Tikztempaltes и их редактирование}

% \begin{figure}[h]
% \begin{center}
%  Рисунок
% \end{center}
% \caption{Стимулирующая монетарная политика в открытой экономике}
% \end{figure}

\subsection{Geogebra}

% Нарисовать что-нибудь рандомное!
% Отдать им одну из глав в TeX!


\end{document} % конец документа